\documentclass{article}
\usepackage[utf8]{inputenc}
\usepackage[T1]{fontenc}
\usepackage{amsmath, amssymb}
\usepackage{geometry}
\usepackage{listings}
\usepackage{xcolor}
\usepackage{hyperref}
\usepackage{ctex}
% 页面设置
\geometry{a4paper, margin=1in}

% 代码块样式设置
\definecolor{codegreen}{rgb}{0,0.6,0}
\definecolor{codegray}{rgb}{0.5,0.5,0.5}
\definecolor{codepurple}{rgb}{0.58,0,0.82}
\definecolor{backcolour}{rgb}{0.95,0.95,0.92}

\lstdefinestyle{mystyle}{
    backgroundcolor=\color{backcolour},
    commentstyle=\color{codegreen},
    keywordstyle=\color{magenta},
    numberstyle=\tiny\color{codegray},
    stringstyle=\color{codepurple},
    basicstyle=\ttfamily\footnotesize,
    breakatwhitespace=false,
    breaklines=true,
    captionpos=b,
    keepspaces=true,
    numbers=left,
    numbersep=5pt,
    showspaces=false,
    showstringspaces=false,
    showtabs=false,
    tabsize=2
}

\lstset{style=mystyle}

\title{QTA 第三周笔试题解答}
\author{欧岱松}
\date{\today}

\begin{document}

\maketitle

\section*{1. 骰子问题}
\textbf{题目:}使用一个标准的 $1 \sim 6$ 骰子,构造出 $1 \sim 7$ 的均匀分布。

\noindent\textbf{解答:}使用拒绝抽样法(Rejection Sampling)。
\begin{enumerate}
    \item 投掷骰子两次,得到结果 $a$ 和 $b$($1 \le a, b \le 6$)。
    \item 计算索引值 $val = (a-1) \times 6 + b$。这将生成 $[1, 36]$ 区间内的均匀整数。
    \item 如果 $val \le 35$:结果为 $(val - 1) \bmod 7 + 1$。
    \item 如果 $val = 36$:舍弃该结果,从第1步重新开始。
\end{enumerate}
由于 35 是 7 的倍数,每个数字 1 到 7 在接受条件下出现的概率均为 $5/35 = 1/7$,因此该方法生成的分布是均匀的。

\section*{2. 截面问题}
\textbf{题目:}圆锥的截面是什么形状?

\noindent\textbf{解答:}
假设切面平行于底面,截面形状为\textbf{圆形}。

\begin{itemize}
    \item \textbf{证明:}设圆锥沿 z 轴对齐,其方程为 $x^2 + y^2 = c^2 z^2$。平行于底面的截面意味着 $z = k$(常数)。将 $k$ 代入方程,得到 $x^2 + y^2 = (ck)^2 = R^2$,这描述的是一个圆。

    \item \textbf{注:}如果平面倾斜,截面将产生圆锥曲线,可能是椭圆、抛物线或双曲线,证明暂时没有写出来。
\end{itemize}

\section*{3. 数字序列问题}
\textbf{题目:}能否将数字 $1,1, \dots, 2026,2026$ 排列,使得两个 $k$ 之间恰好有 $k$ 个数字?

\noindent\textbf{解答:}\textbf{不能。}

这是一个 Langford 配对问题。对于数字 $n$,存在这样排列的必要条件是:
$$ n \equiv 0 \pmod 4 \quad \text{或} \quad n \equiv 3 \pmod 4 $$

检验 $n = 2026$:
$$ 2026 = 4 \times 506 + 2 \implies 2026 \equiv 2 \pmod 4 $$

由于不满足必要条件,因此这样的排列不存在。

\textbf{必要条件的证明:}

假设 Langford 序列长度为 $2n$,所有位置下标 $\{1, 2, \dots, 2n\}$ 的总和为:
$$ S = \sum_{i=1}^{2n} i = n(2n+1) = 2n^2 + n $$

对于数字 $k$,设其两个位置分别为 $x_k$ 和 $y_k$,由题意有 $y_k = x_k + k + 1$(两个 $k$ 之间恰好有 $k$ 个数字)。

因此,数字 $k$ 的两个位置之和为:
$$ x_k + y_k = x_k + (x_k + k + 1) = 2x_k + k + 1 $$

对所有 $k \in \{1, 2, \dots, n\}$ 求和,得到位置总和的另一种表达:
$$ S = \sum_{k=1}^{n} (2x_k + k + 1) = 2\sum_{k=1}^{n} x_k + \sum_{k=1}^{n} k + \sum_{k=1}^{n} 1 = 2\sum_{k=1}^{n} x_k + \frac{n(n+1)}{2} + n $$

联立两个 $S$ 的表达式:
$$ 2n^2 + n = 2\sum_{k=1}^{n} x_k + \frac{n(n+1)}{2} + n $$

化简得:
$$ 2n^2 = 2\sum_{k=1}^{n} x_k + \frac{n(n+1)}{2} $$

移项:
$$ 2\left(n^2 - \sum_{k=1}^{n} x_k\right) = \frac{n(n+1)}{2} $$

由于左边显然为偶数,因此右边的 $\frac{n(n+1)}{2}$ 也必须是偶数,即 $n(n+1)$ 必须能被 4 整除。

由于 $n$ 和 $n+1$ 是连续整数,其中必有一个是偶数。要使 $n(n+1) \equiv 0 \pmod{4}$,需要:
\begin{itemize}
    \item 若 $n$ 是偶数,则 $n \equiv 0 \pmod{4}$
    \item 若 $n$ 是奇数,则 $n+1$ 必须被 4 整除,即 $n \equiv 3 \pmod{4}$
\end{itemize}

因此,Langford 序列存在的必要条件为:
$$ n \equiv 0 \pmod{4} \quad \text{或} \quad n \equiv 3 \pmod{4} $$

\section*{4. 蓝眼睛问题}
\textbf{题目:}一个岛上有 100 个蓝眼睛的人和 900 个棕眼睛的人。岛上的规则是:如果有人知道自己是蓝眼睛,就必须在当晚自杀。某天,一个外来者说:"我看到了一个蓝眼睛的人。"会发生什么?

\noindent\textbf{解答:}
\textbf{第 100 天晚上,所有 100 个蓝眼睛的人都会自杀。}

\textbf{数学归纳法:}
    \begin{itemize}
        \item \textbf{基础情况:}如果只有 1 个蓝眼睛的人,他听到外来者的话后,看到其他人都是棕眼睛,就知道自己是蓝眼睛,第 1 天晚上自杀。
        \item \textbf{归纳步骤:}假设有 $N$ 个蓝眼睛的人。每个蓝眼睛的人看到其他 $N-1$ 个蓝眼睛的人。如果前 $N-1$ 天没有人自杀,他们就能推断出蓝眼睛的人数不是 $N-1$,而是 $N$(包括自己),因此第 $N$ 天晚上所有蓝眼睛的人都会自杀。
    \end{itemize}


\section*{5. 猜牌问题}
\textbf{题目:}Bob 从一副 52 张牌中抽取 5 张,向 Alice 展示其中 4 张。Alice 能否根据这 4 张牌推断出第 5 张牌?

\noindent\textbf{解答:}\textbf{可以。}

\textbf{策略:}
\begin{enumerate}
    \item \textbf{确定花色(鸽笼原理):}5 张牌中至少有 2 张花色相同。Bob 选择其中一张作为隐藏牌,另一张作为第一张展示的牌,以此传递花色信息。

    \item \textbf{编码点数:}剩余 3 张牌可以按 $3! = 6$ 种不同方式排列。Bob 计算隐藏牌与展示牌之间的距离 $d$(模 13),选择使 $d \le 6$ 的方向。然后用 3 张牌的排列顺序编码这个距离 $d$。Alice 根据后 3 张牌的排列顺序解码出距离 $d$,将第一张牌的点数加上 $d$(模 13)即可得到隐藏牌。
\end{enumerate}


\section*{6. 盛最多水的容器}
\textbf{题目:}给定一个整数数组 $height$,其中 $height[i]$ 表示第 $i$ 条垂直线的高度。找出两条线,使得它们与 x 轴构成的容器能盛最多的水。

\noindent\textbf{解答:}使用\textbf{双指针算法},时间复杂度为 $O(N)$。

\textbf{算法思路:}
\begin{enumerate}
    \item 初始化左右指针分别指向数组的起始和末尾。
    \item 计算当前容器的面积 = $\min(height[left], height[right]) \times (right - left)$。
    \item 更新最大面积。
    \item 移动较短的那一边的指针(贪心策略)。
    \item 重复步骤 2-4,直到两指针相遇。
\end{enumerate}
\textbf{代码实现:}

\begin{lstlisting}[language=Python, caption=双指针算法 Python 实现]
def maxArea(height):
    """
    求解盛最多水的容器问题
    参数: height - 表示每条垂直线高度的整数数组
    返回: 容器能够容纳的最大水量
    """
    # 初始化左右指针
    left, right = 0, len(height) - 1
    max_area = 0

    # 当两指针未相遇时循环
    while left < right:
        # 计算当前容器面积
        current_height = min(height[left], height[right])
        current_width = right - left
        current_area = current_height * current_width

        # 更新最大面积
        max_area = max(max_area, current_area)

        # 贪心策略:移动较短的一边
        if height[left] < height[right]:
            left += 1
        else:
            right -= 1

    return max_area


# 测试示例
height = [1, 8, 6, 2, 5, 4, 8, 3, 7]
result = maxArea(height)
print("输入:", height)
print("输出:", result)
# 解释: 选择 height[1]=8 和 height[8]=7
#       面积 = min(8,7) * (8-1) = 7 * 7 = 49
\end{lstlisting}


\end{document}
