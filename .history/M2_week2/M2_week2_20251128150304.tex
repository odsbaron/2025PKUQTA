\documentclass[11pt, a4paper]{ctexart}
\usepackage[utf8]{inputenc}
\usepackage[T1]{fontenc}
\usepackage{amsmath, amssymb, amsthm}
\usepackage{geometry}
\usepackage{enumitem}
\usepackage{tcolorbox}
\usepackage{fancyhdr}

% 页面设置
\geometry{left=2.5cm, right=2.5cm, top=2.5cm, bottom=2.5cm}
\setlength{\parindent}{0pt}
\setlength{\parskip}{0.8em}

% 定义数学符号
\newcommand{\E}{\mathbb{E}}
\newcommand{\Prob}{\mathbb{P}}

% 标题信息
\title{\textbf{面试实战案例:伯努利流中的序列模式延迟与套利}}
\author{Quant Researcher Interview Case Study}
\date{}

\begin{document}

\maketitle

% 题目背景框
\begin{tcolorbox}[colback=gray!5!white, colframe=blue, title=\textbf{题目背景}]
在金融高频交易中,我们监控一个二元信号流 $S = \{x_1, x_2, \dots\}$,其中 $x_i$ 独立服从伯努利分布 $x_i \sim \text{Bernoulli}(0.5)$(即 $P(H)=P(T)=0.5$)。

\end{tcolorbox}

\vspace{1em}

% Part 1
\section*{Part 1}
假设我们需要检测长度为 2 的短信号。我们定义以下随机变量:
\begin{itemize}
    \item $N_{HH}$:序列中首次连续出现两个“正面”(HH)所需的观测次数。
    \item $N_{HT}$:序列中首次出现“正面”紧接“反面”(HT)所需的观测次数。
\end{itemize}

\begin{enumerate}[label=\textbf{问题 1.\arabic*:}, leftmargin=3.5em]
    \item 请分别计算期望值 $\E[N_{HH}]$ 和 $\E[N_{HT}]$。
\end{enumerate}

\vspace{1em}
\hrule
\vspace{1em}

% Part 2
\section*{Part 2: 广义模式推导 }
现在考虑一个长度为 $L$ 的任意目标序列 pattern $A = (a_1, a_2, \dots, a_L)$。为了求解 $\E[N_A]$,我们引入一个\textbf{公平赌场模型(Fair Casino Model)}:

\begin{quote}
    假设每一时刻 $t=1, 2, \dots$ 都有一个新的赌徒进场。\\
    第 $t$ 个赌徒在进场时押注 1 美元,赌 $x_t = a_1$。如果是公平赌局(赔率为 1:1),赢了变成 2 美元,输了变成 0。\\
    如果赢了,他将 2 美元全押在下一时刻 $x_{t+1} = a_2$ 上,以此类推。\\
    一旦输掉任何一次,赌徒破产离开。如果连续赢了 $L$ 次(即匹配了整个序列 $A$),赌场关闭。
\end{quote}

\begin{tcolorbox}[colback=yellow!10!white, colframe=orange!75!black, title=\textbf{提示 (Hint)}]
利用鞅(Martingale)的性质,可以证明对于任意长度为 $L$ 的二进制序列 $A$,其首次出现的期望时间公式为:
\begin{equation*}
    \E[N_A] = \sum_{k=1}^{L} \delta_k \cdot 2^k
\end{equation*}
其中 $\delta_k$ 是一个指示函数:当序列 $A$ 的长度为 $k$ 的\textbf{前缀}同时也是它的\textbf{后缀}时,$\delta_k=1$,否则 $\delta_k=0$。
\end{tcolorbox}

\begin{enumerate}[label=\textbf{问题 2.\arabic*:}, leftmargin=3.5em]
    \item 利用上述公式,快速计算模式 \textbf{HTHH} 的期望等待时间。
\end{enumerate}

\vspace{1em}
\hrule
\vspace{1em}

% Part 3
\section*{Part 3: 彭尼博弈 (Penney's Game - Competitive Strategy)}
现在有两个交易员 Alice 和 Bob 分别选择不同的模式进行“抢跑”游戏。一旦信号流中出现了某人的模式,该人获胜,游戏结束。
\begin{itemize}
    \item Alice 选择了模式 $A = \text{H H}$。
    \item Bob 选择了模式 $B = \text{T H}$。
\end{itemize}

\begin{enumerate}[label=\textbf{问题 3}, leftmargin=3.5em]
    \item 请问 Bob 获胜的概率(即 $B$ 先于 $A$ 出现的概率)是多少?(注意,这里不再是计算“期望时间”,而是计算\textbf{获胜概率}。)
    
   % \item 如果 Alice 选了 $HHH$,Bob 想要在这个非传递性游戏中占据优势(即获胜概率 $>0.5$),他应该选择什么长度为 3 的模式?\\
   % \textit{(请直接给出结论,无需证明。)}
\end{enumerate}

\end{document}