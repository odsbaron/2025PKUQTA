\documentclass[11pt, a4paper]{ctexart}
\usepackage[utf8]{inputenc}
\usepackage[T1]{fontenc}
\usepackage{amsmath, amssymb, amsthm}
\usepackage{enumitem}
\usepackage{geometry}
\usepackage{listings}
\usepackage{xcolor}
\geometry{left=2.5cm, right=2.5cm, top=3cm, bottom=3cm}

\definecolor{codebg}{RGB}{245, 245, 245}
\definecolor{codegreen}{RGB}{39, 174, 96}
\definecolor{codegray}{RGB}{127, 140, 141}
\definecolor{codepurple}{RGB}{142, 68, 173}

\lstset{
    language=Python,
    backgroundcolor=\color{codebg},
    commentstyle=\color{codegreen},
    keywordstyle=\color{magenta},
    numberstyle=\tiny\color{codegray},
    stringstyle=\color{codepurple},
    basicstyle=\ttfamily\footnotesize,
    breakatwhitespace=false,
    breaklines=true,
    captionpos=b,
    keepspaces=true,
    numbers=left,
    numbersep=5pt,
    showspaces=false,
    showstringspaces=false,
    showtabs=false,
    tabsize=4
}

\setlength{\parindent}{0pt}
\setlength{\parskip}{0.6em}

% 标题设置
\title{\textbf{QTA笔面试刷题week1-答案}}
\date{}

\begin{document}

\maketitle

\begin{enumerate}[label=\arabic*., itemsep=2.5em]

    % 题目 1
    \item \textbf{简单的抽球问题}

    \textbf{题目:}一个袋子里有4个红球、3个蓝球和2个黄球。随机不放回抽3个球,请问抽到的3个球至少有两种不同颜色的概率有多少?

    \textbf{解法:}补集法。总共 $\binom{9}{3} = 84$ 种抽法。只有一种颜色:3个红球 $\binom{4}{3} = 4$ 种,3个蓝球 $\binom{3}{3} = 1$ 种,共5种。

    \textbf{答案:}$\displaystyle P = 1 - \frac{5}{84} = \frac{79}{84} $

    % 题目 2
    \item \textbf{切线段的极限和}

    \textbf{题目:}一条长度为1的线段,随机在中间切一刀,得到两部分$x$和$y$。把它们相乘并加到和里。再把$x$和$y$各自随机切一刀,如此重复操作。请问这个和的极限值是多少?

    \textbf{解法:}
    设 $E(L)$ 为长度为 $L$ 的线段切分后的乘积和期望值。根据递归定义,总期望等于当前切分收益 $x(L-x)$ 加上剩余两段的期望收益在 $[0, L]$ 上的平均值,即满足方程:
\begin{equation}
    E(L) = \frac{1}{L} \int_0^L \left[ x(L-x) + E(x) + E(L-x) \right] dx
\end{equation}
根据量纲分析设 $E(L) = k L^2$,利用积分对称性将通解代入方程,可得:
\[
    k L^2 = \frac{1}{L} \left[ \int_0^L (xL - x^2) dx + 2 \int_0^L k x^2 dx \right]
\]
计算积分项得 $\frac{L^3}{6}$ 与 $\frac{2kL^3}{3}$,代入化简得 $k = \frac{1}{6} + \frac{2}{3}k$,解得 $k = \frac{1}{2}$。因此当 $L=1$ 时,极限值为 $E(1) = \frac{1}{2}$。
    \item \textbf{年久失修的密码锁}

    \textbf{题目:}一个3位的密码锁,年久失修功能异常,只需要输入任意两位密码就能打开。最少需要尝试多少次?

    \textbf{解法:}
    针对三位密码锁只需匹配任意两位即可打开的,其尝试次数的实际下界分析需从理论体积极限出发。
    首先,由于总状态空间为 $10^3$,而单次尝试,能覆盖自身及三个维度上各改变一位的 $9$ 个状态,
    单次总覆盖体积为 $1+9+9+9=28$;理论上若假设这些覆盖区域完美填充空间,则所需最小次数为 $1000/28 \approx 35.71$,即至少 $36$ 次。
    然而,因为在 $10 \times 10 \times 10$ 的离散网格中,固定的“十字形”覆盖域无法在不发生重叠的情况下完美拼合。所以需要可以找到一些实际可以构造的下界。
    
    \section{构造定义 (Construction Definition)}

为了实现对 $10 \times 10 \times 10$ 解空间的 \textbf{100\% 确定性覆盖},我们构造一个包含 $N=60$ 次尝试的集合 $C$。该集合由 6 个子集 $S_k$ 并联而成。

令 $C = \bigcup_{k \in K} S_k$,其中关键参数集合 $K$ 定义为:
\begin{equation}
    K = \{ 0, 1, 2, 3, 5, 7 \}
\end{equation}

对于任意 $k \in K$,子集 $S_k$ 包含 10 个码字,其元素 $(x, y, z)$ 由首位数字 $x$ 遍历 $\{0, 1, \dots, 9\}$ 并满足以下同余方程组生成:
\begin{align}
    y &\equiv (x + 2k) \pmod{10} \\
    z &\equiv (x + k) \phantom{2} \pmod{10}
\end{align}

这相当于在三维离散空间中选取了 6 条具有特定斜率和偏移量的“空间对角线”。

\section{完备性证明 (Proof of Completeness)}

对于任意目标密码 $P=(a, b, c)$,只要尝试集合能覆盖其三个维度投影差值 $b-a$、$c-a$ 或 $c-b$ 中的任意一个,锁即可打开。我们验证集合 $K$ 在三个维度上生成的差值覆盖能力:

\begin{table}[h]
\centering
\caption{参数 $k$ 在不同维度上的差值覆盖表}
\vspace{0.2cm}
\begin{tabular}{c c c c}
\toprule
\textbf{参数 $k \in K$} & \textbf{XY平面投影 ($2k$)} & \textbf{XZ平面投影 ($k$)} & \textbf{YZ平面投影 ($-k$)} \\
 & \small{(覆盖 $y-x$)} & \small{(覆盖 $z-x$)} & \small{(覆盖 $z-y$)} \\
\midrule
0 & 0 & 0 & 0 \\
1 & 2 & 1 & 9 \\
2 & 4 & 2 & 8 \\
3 & 6 & 3 & 7 \\
5 & 0 & 5 & 5 \\
7 & 4 & 7 & 3 \\
\midrule
\textbf{覆盖集合} & $\{0, 2, 4, 6\}$ & $\{1, 3, 5, 7\}$ & $\{8, 9\}$ \\
\bottomrule
\end{tabular}
\end{table}

\textbf{覆盖逻辑分析:}
\begin{itemize}
    \item \textbf{XY平面 (偶数覆盖):} 差值 $2k \pmod{10}$ 成功覆盖了所有小偶数 $\{0, 2, 4, 6\}$。
    \item \textbf{XZ平面 (奇数覆盖):} 差值 $k \pmod{10}$ 成功覆盖了所有小奇数 $\{1, 3, 5, 7\}$。
    \item \textbf{YZ平面 (高位补漏):} 差值 $-k \pmod{10}$ 成功覆盖了剩余的高位差值 $\{8, 9\}$。
\end{itemize}

\textbf{结论:}
三个覆盖集合的并集为:
\[
\{0, 2, 4, 6\} \cup \{1, 3, 5, 7\} \cup \{8, 9\} = \{0, 1, 2, 3, 4, 5, 6, 7, 8, 9\}
\]
这证明了无论密码数字之间的相对距离为何,该 60 次构造均能保证至少在一个投影面上与目标密码重合,从而实现完美覆盖。
    % 题目 4
    \item \textbf{把守圆圈}

    \textbf{题目:}有一个100米半径的圆。超人站在圆心,速度为1米/秒。蝙蝠侠站在圆周上某一点。问:蝙蝠侠速度最低是什么值,才能使得超人无法逃出这个圆?

    \textbf{解法:}狮子与人问题。超人最优策略分两阶段:

    (1) 螺旋移动到半径 $r = \frac{100}{v_b}$ 处,使双方角速度相等

    (2) 从半径 $r$ 处直线冲向圆周(选择与蝙蝠侠相反方向)

    临界条件:超人冲刺距离 $100 - \frac{100}{v_b}$ = 蝙蝠侠跑半圆时间 $\frac{100\pi}{v_b}$

    解得:$v_b = 1 + \pi$

    \textbf{答案:}$\displaystyle v_b = 1 + \pi \approx 4.14$ 米/秒

    % 题目 5
    \item \textbf{快速排序的性质(多选)}

    \textbf{题目:}以下是快速排序的特性有:

    A. 第一趟排序后,所有元素都在其最终位置上

    B. 最坏情况时间复杂度为$O(n^2)$

    C. 是不稳定的排序算法

    D. 空间复杂度为$O(\log n)$到$O(n)$之间

    \textbf{解法:}

    A错误:只有pivot元素在最终位置,其他元素不一定

    B正确:当每次pivot都是极值时(如已排序数组),递归深度为$n$,时间$O(n^2)$

    C正确:长距离交换可能改变相同元素的相对顺序,因此不稳定

    D正确:递归调用栈深度:最好$O(\log n)$(平衡),最坏$O(n)$(极度不平衡)

    \textbf{答案:BCD}

    % 题目 6
    \item \textbf{Expected Distinct Numbers}

    \textbf{题目:}Say you have $n$ numbers $\{1, 2, \ldots, n\}$, and you uniformly sample from this distribution with replacement $n$ times. What is the expected number of distinct values you would draw?

    \textbf{解法:}指示随机变量法。定义 $X_i = 1$ 当且仅当数字 $i$ 至少出现一次。不同数字总数 $X = \sum_{i=1}^{n} X_i$。

    计算:$P(X_i = 1) = 1 - \left(1 - \frac{1}{n}\right)^n$

    由期望的线性性:$\mathbb{E}[X] = n \cdot \left[1 - \left(1 - \frac{1}{n}\right)^n\right]$

    当 $n \to \infty$ 时,$\left(1 - \frac{1}{n}\right)^n \to \frac{1}{e}$

    \textbf{答案:}$\displaystyle n \left[1 - \left(1 - \frac{1}{n}\right)^n\right] \approx 0.632n$

    % 题目 7
    \item \textbf{Arithmetic Subarrays}

    \textbf{题目:}Given an integer array \texttt{nums}, return the number of arithmetic subarrays of \texttt{nums}. (等差子数组:至少3个元素,相邻元素差相等)

    \textbf{解法:}动态规划。定义 $dp[i]$ = 以 $\texttt{nums}[i]$ 结尾的等差子数组个数

    状态转移:
    \[
    dp[i] = \begin{cases}
    dp[i-1] + 1, & \text{if } \texttt{nums}[i] - \texttt{nums}[i-1] = \texttt{nums}[i-1] - \texttt{nums}[i-2] \\
    0, & \text{otherwise}
    \end{cases}
    \]


    \textbf{Python代码:}

    \begin{lstlisting}
def numberOfArithmeticSlices(nums):
    n = len(nums)
    if n < 3:
        return 0

    total = 0
    dp = 0

    for i in range(2, n):
        if nums[i] - nums[i-1] == nums[i-1] - nums[i-2]:
            dp += 1
            total += dp
        else:
            dp = 0

    return total
    \end{lstlisting}

\end{enumerate}

\vspace{2em}
\hrule
\vspace{1em}


\end{document}
