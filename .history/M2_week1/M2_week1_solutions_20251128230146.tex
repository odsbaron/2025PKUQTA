\documentclass[11pt, a4paper]{ctexart}
\usepackage[utf8]{inputenc}
\usepackage[T1]{fontenc}
\usepackage{amsmath, amssymb, amsthm}
\usepackage{enumitem}
\usepackage{geometry}
\usepackage{tcolorbox}
\usepackage{xcolor}
\geometry{left=2.5cm, right=2.5cm, top=3cm, bottom=3cm}

% 统一颜色方案
\definecolor{primarydark}{RGB}{41, 128, 185}
\definecolor{primarylight}{RGB}{174, 214, 241}
\definecolor{secondarydark}{RGB}{52, 73, 94}
\definecolor{secondarylight}{RGB}{236, 240, 241}
\definecolor{accentdark}{RGB}{230, 126, 34}
\definecolor{accentlight}{RGB}{255, 242, 230}
\definecolor{successdark}{RGB}{39, 174, 96}
\definecolor{successlight}{RGB}{232, 248, 239}

\setlength{\parindent}{0pt}
\setlength{\parskip}{0.6em}

% 标题设置
\title{\textbf{QTA笔面试刷题week1-答案}}
\date{}

\begin{document}

\maketitle

\begin{enumerate}[label=\arabic*., itemsep=2.5em]

    % 题目 1
    \item \textbf{简单的抽球问题}

    \begin{tcolorbox}[colback=secondarylight, colframe=secondarydark, title=\textbf{题目}]
    一个袋子里有4个红球、3个蓝球和2个黄球。随机不放回抽3个球,请问抽到的3个球至少有两种不同颜色的概率有多少?
    \end{tcolorbox}

    \textbf{解法:补集法}

    总共有 $\binom{9}{3} = 84$ 种抽法。

    "至少两种颜色"的反面是"只有一种颜色":
    \begin{itemize}[leftmargin=2em, itemsep=0.2em]
        \item 3个红球:$\binom{4}{3} = 4$
        \item 3个蓝球:$\binom{3}{3} = 1$
        \item 3个黄球:不可能(只有2个)
    \end{itemize}

    概率:$P = 1 - \frac{5}{84} = \frac{79}{84} \approx 94.05\%$

    \begin{tcolorbox}[colback=successlight, colframe=successdark]
    \textbf{答案:}$\displaystyle \frac{79}{84}$
    \end{tcolorbox}

    % 题目 2
    \item \textbf{切线段的极限和}

    \begin{tcolorbox}[colback=secondarylight, colframe=secondarydark, title=\textbf{题目}]
    一条长度为1的线段,随机在中间切一刀,得到两部分$x$和$y$。把它们相乘并加到和里:$\text{sum} = x \cdot y$。再把$x$和$y$各自随机切一刀...如此重复操作。请问这个和的极限值是多少?
    \end{tcolorbox}

    \textbf{解法:递推关系}

    设 $S(L)$ 为长度 $L$ 的线段经过无限次切割后的期望和。

    \textbf{递推式:}长度为 $L$ 的线段随机切开,两段乘积的期望为 $\frac{L^2}{6}$(积分计算)。设 $S(L) = cL^2$(猜测形式),则:
    \[
    S(L) = \frac{L^2}{6} + 2\mathbb{E}[S(t)] = \frac{L^2}{6} + 2c \cdot \frac{L^2}{3}
    \]

    由 $S(L) = cL^2$,得:$c = \frac{1}{6} + \frac{2c}{3}$,解得 $c = \frac{1}{2}$

    \begin{tcolorbox}[colback=successlight, colframe=successdark]
    \textbf{答案:}$\displaystyle S(1) = \frac{1}{2}$
    \end{tcolorbox}

    % 题目 3
    \item \textbf{年久失修的密码锁}

    \begin{tcolorbox}[colback=secondarylight, colframe=secondarydark, title=\textbf{题目}]
    一个3位的密码锁,年久失修功能异常,只需要输入任意两位密码就能打开。最少需要尝试多少次?
    \end{tcolorbox}

    \textbf{解法:De Bruijn序列}

    需要构造包含所有 $10^2 = 100$ 种两位数组合的最短序列。

    \textbf{图论建模:}
    \begin{itemize}[leftmargin=2em, itemsep=0.2em]
        \item 顶点:10个数字 $\{0, 1, \ldots, 9\}$
        \item 有向边:从 $i$ 到 $j$ 代表两位数 $ij$,共100条边
        \item 问题等价于找欧拉回路(每条边恰好经过一次)
    \end{itemize}

    欧拉回路长度 = 边数 = 100,因此最少需要100次尝试。

    \begin{tcolorbox}[colback=successlight, colframe=successdark]
    \textbf{答案:}\textbf{100次}
    \end{tcolorbox}

    % 题目 4
    \item \textbf{把守圆圈}

    \begin{tcolorbox}[colback=secondarylight, colframe=secondarydark, title=\textbf{题目}]
    有一个100米半径的圆。超人站在圆心,速度为1米/秒。蝙蝠侠站在圆周上某一点。问:蝙蝠侠速度最低是什么值,才能使得超人无法逃出这个圆?
    \end{tcolorbox}

    \textbf{解法:狮子与人问题}

    超人最优策略分两阶段:

    \textbf{阶段1:}螺旋移动到半径 $r = \frac{R}{v_b} = \frac{100}{v_b}$ 处,此时双方角速度相等
    \begin{itemize}[leftmargin=2em, itemsep=0.2em]
        \item 超人角速度:$\omega_s = \frac{1}{r}$
        \item 蝙蝠侠角速度:$\omega_b = \frac{v_b}{100}$
        \item 相等条件:$r = \frac{100}{v_b}$
    \end{itemize}

    \textbf{阶段2:}超人从半径 $r$ 处直线冲向圆周(选择与蝙蝠侠相反方向)
    \begin{itemize}[leftmargin=2em, itemsep=0.2em]
        \item 超人冲刺距离:$100 - \frac{100}{v_b}$,耗时相同
        \item 蝙蝠侠跑半圆:$100\pi$,耗时 $\frac{100\pi}{v_b}$
    \end{itemize}

    \textbf{临界条件:}$100 - \frac{100}{v_b} = \frac{100\pi}{v_b}$

    解得:$v_b = 1 + \pi$

    \begin{tcolorbox}[colback=successlight, colframe=successdark]
    \textbf{答案:}$\displaystyle v_b = 1 + \pi \approx 4.14$ 米/秒
    \end{tcolorbox}

    % 题目 5
    \item \textbf{快速排序的性质}

    \begin{tcolorbox}[colback=secondarylight, colframe=secondarydark, title=\textbf{题目(多选)}]
    以下是快速排序的特性有:

    \textbf{A.} 第一趟排序后,所有元素都在其最终位置上

    \textbf{B.} 最坏情况时间复杂度为$O(n^2)$

    \textbf{C.} 是不稳定的排序算法

    \textbf{D.} 空间复杂度为$O(\log n)$到$O(n)$之间
    \end{tcolorbox}

    \textbf{逐项分析:}

    \textbf{A. 错误} — 只有pivot元素在最终位置,其他元素不一定

    \textbf{B. 正确} — 当每次pivot都是极值时(如已排序数组),递归深度为$n$,时间$O(n^2)$

    \textbf{C. 正确} — 长距离交换可能改变相同元素的相对顺序,因此不稳定

    \textbf{D. 正确} — 递归调用栈深度:最好$O(\log n)$(平衡),最坏$O(n)$(极度不平衡)

    \begin{tcolorbox}[colback=successlight, colframe=successdark]
    \textbf{答案:BCD}
    \end{tcolorbox}

    % 题目 6
    \item \textbf{Expected Distinct Numbers}

    \begin{tcolorbox}[colback=secondarylight, colframe=secondarydark, title=\textbf{题目}]
    Say you have $n$ numbers $\{1, 2, \ldots, n\}$, and you uniformly sample from this distribution \textbf{with replacement} $n$ times. What is the expected number of distinct values you would draw?
    \end{tcolorbox}

    \textbf{解法:指示随机变量法}

    定义指示变量:$X_i = 1$ 当且仅当数字 $i$ 至少出现一次

    不同数字总数:$X = \sum_{i=1}^{n} X_i$

    \textbf{计算 $P(X_i = 1)$:}
    \begin{align*}
    P(X_i = 1) &= 1 - P(\text{数字}i\text{从未出现}) \\
    &= 1 - \left(1 - \frac{1}{n}\right)^n
    \end{align*}

    由期望的线性性:
    \[
    \mathbb{E}[X] = n \cdot \left[1 - \left(1 - \frac{1}{n}\right)^n\right]
    \]

    当 $n \to \infty$ 时,$\left(1 - \frac{1}{n}\right)^n \to \frac{1}{e}$,因此:
    \[
    \mathbb{E}[X] \approx n\left(1 - \frac{1}{e}\right) \approx 0.632n
    \]

    \begin{tcolorbox}[colback=successlight, colframe=successdark]
    \textbf{答案:}$\displaystyle n \left[1 - \left(1 - \frac{1}{n}\right)^n\right] \approx 0.632n$
    \end{tcolorbox}

    % 题目 7
    \item \textbf{Arithmetic Subarrays}

    \begin{tcolorbox}[colback=secondarylight, colframe=secondarydark, title=\textbf{题目}]
    Given an integer array \texttt{nums}, return the number of arithmetic subarrays of \texttt{nums}.

    (等差子数组:至少3个元素,相邻元素差相等)
    \end{tcolorbox}

    \textbf{解法:动态规划}

    定义 $dp[i]$ = 以 $\texttt{nums}[i]$ 结尾的等差子数组个数

    \textbf{状态转移:}
    \[
    dp[i] = \begin{cases}
    dp[i-1] + 1, & \text{if } \texttt{nums}[i] - \texttt{nums}[i-1] = \texttt{nums}[i-1] - \texttt{nums}[i-2] \\
    0, & \text{otherwise}
    \end{cases}
    \]

    \textbf{举例:}\texttt{nums = [1, 2, 3, 4]}
    \begin{itemize}[leftmargin=2em, itemsep=0.2em]
        \item $dp[2] = 1$:新增 [1,2,3]
        \item $dp[3] = 2$:新增 [2,3,4] 和 [1,2,3,4]
        \item 总数 = $1 + 2 = 3$
    \end{itemize}

    \textbf{最终答案:}$\sum_{i=2}^{n-1} dp[i]$,时间 $O(n)$,空间 $O(1)$

    \begin{tcolorbox}[colback=successlight, colframe=successdark]
    \textbf{答案:}使用DP,$dp[i] = dp[i-1] + 1$ 或 $0$,累加求和
    \end{tcolorbox}

\end{enumerate}

\vspace{2em}
\hrule
\vspace{1em}


\end{document}
