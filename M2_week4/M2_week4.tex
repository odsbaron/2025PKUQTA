% !TEX program = xelatex
% !TEX encoding = UTF-8

\documentclass[12pt, a4paper]{article}

% --- 宏包配置 ---
\usepackage[UTF8]{ctex}           % 中文支持
\usepackage{geometry}             % 页面布局
\usepackage{amsmath, amssymb}     % 数学公式
\usepackage{graphicx}             % 图片支持
\usepackage{xcolor}               % 颜色支持
\usepackage{listings}             % 代码高亮
\usepackage{tikz}                 % 绘图支持
\usepackage{fancyhdr}             % 页眉页脚

% --- 页面设置 ---
\geometry{left=2.5cm, right=2.5cm, top=2.5cm, bottom=2.5cm}
\setlength{\headheight}{14.5pt}
\pagestyle{fancy}
\fancyhf{}
\lhead{Week 4 QTA 笔试题解}
\rhead{\today}
\cfoot{\thepage}

% --- 代码块样式定义 ---
\definecolor{codegreen}{rgb}{0,0.6,0}
\definecolor{codegray}{rgb}{0.5,0.5,0.5}
\definecolor{codepurple}{rgb}{0.58,0,0.82}
\definecolor{backcolour}{rgb}{0.96,0.96,0.96}

\lstdefinestyle{pystyle}{
    backgroundcolor=\color{backcolour},   
    commentstyle=\color{codegreen},
    keywordstyle=\color{magenta}\bfseries,
    numberstyle=\tiny\color{codegray},
    stringstyle=\color{codepurple},
    basicstyle=\ttfamily\small,
    breakatwhitespace=false,         
    breaklines=true,                 
    captionpos=b,                    
    keepspaces=true,                 
    numbers=left,                    
    numbersep=5pt,                  
    showspaces=false,                
    showstringspaces=false,
    showtabs=false,                  
    tabsize=4,
    frame=single,
    rulecolor=\color{black!20}
}
\lstset{style=pystyle}

% --- 文档开始 ---
\begin{document}

\title{\textbf{Week 4 QTA 笔试题标准题解}}
\author{欧岱松}
\date{\today}
\maketitle

\section*{1. 必胜策略 I (博弈论)}

\subsubsection*{解答}
\textbf{结论:先手必胜。}

\textbf{答案:}
先手玩家首先取走最中间的 \textbf{1} 个积木,将 31 个积木分解为左右两段完全独立的部分。

每段各 15 个,此时局面处于完全对称状态。此后,无论后手玩家在左边的一堆进行什么操作(取走 $x$ 个,或将堆切分),先手玩家只需在右边的一堆执行完全相同的镜像操作。由于局面对称,只要后手玩家能进行合法移动,先手玩家必然能在另一侧进行对应的合法移动,因此最后一个积木一定会被先手玩家拿走。

\vspace{1cm}

\section*{2. 切割木棍 (几何概率)}

\subsubsection*{解答}
\textbf{答案:$1/4$ (25\%)。}

设木棍总长为 1,两个切点位置分别为 $x, y$,且 $x, y \sim U(0,1)$。
不失一般性,设 $x < y$,则三段长度为 $x$,$y-x$,$1-y$。

构成三角形的充要条件是任意两边之和大于第三边,等价于\textbf{任意一边长度 $< \frac{1}{2}$},即:
$$x < \frac{1}{2}, \quad y-x < \frac{1}{2}, \quad 1-y < \frac{1}{2}$$
化简得:$x < \frac{1}{2}$,$y < x + \frac{1}{2}$,$y > \frac{1}{2}$。

由对称性,在 $(x,y)$ 平面上,可行域由两个对称的三角形组成。


在单位正方形面积中,满足条件的区域由两个底和高均为 0.5 的小三角形组成:
$$ S = 2 \times \left( \frac{1}{2} \times 0.5 \times 0.5 \right) = 0.25 $$

\newpage

\section*{3. 球的放法 (排列组合)}

\subsubsection*{解答}
\textbf{答案:256 种。}

\textbf{分步计数:}

设绿盒中放入 $g$ 个球,其中 $0 \le g \le 30$,则剩余 $r = 30 - g$ 个球需要放入两个\textbf{相同}的红盒。设两红盒球数为 $a, b$,满足 $a+b=r$ 且 $a \le b$。对于给定的 $r$,方案数为 $N_r = \lfloor r/2 \rfloor + 1$:当 $r$ 为偶数时,可分配方案为 $(0, r), (1, r-1), \ldots, (r/2, r/2)$,共 $r/2 + 1$ 种;当 $r$ 为奇数时,可分配方案为 $(0, r), (1, r-1), \ldots, (\lfloor r/2 \rfloor, \lceil r/2 \rceil)$,共 $\lfloor r/2 \rfloor + 1$ 种。因此总方案数为 $\text{Sum} = \sum_{r=0}^{30} (\lfloor r/2 \rfloor + 1)$。

\textbf{计算过程:}
\begin{align*}
\text{Sum} &= \underbrace{(1+1)}_{r=0,1} + \underbrace{(2+2)}_{r=2,3} + \underbrace{(3+3)}_{r=4,5} + \cdots + \underbrace{(15+15)}_{r=28,29} + \underbrace{16}_{r=30} \\
    &= 2 \times (1 + 2 + 3 + \cdots + 15) + 16 \\
    &= 2 \times \frac{15 \times 16}{2} + 16 \\
    &= 2 \times 120 + 16 = 256
\end{align*}

\section*{4. 必胜策略 II}

\subsubsection*{解答}
\textbf{结论:蝙蝠侠存在必胜策略。}

\textbf{核心逻辑:}
超人的轨迹完全由初始位置 $x_0$ 和速度 $v$ 决定,位置函数为 $x(t) = x_0 + vt$。
因为 $x_0, v \in \mathbb{Z}$,所有可能的参数对集合 $S = \mathbb{Z} \times \mathbb{Z}$ 是\textbf{可数集}。

\textbf{必胜策略构造:}

首先将所有可能的整数对 $(x_0, v)$ 按对角线法排列成序列 $P_1, P_2, P_3, \ldots$。在第 $t$ 时刻,蝙蝠侠假设超人参数为序列中第 $t$ 个参数对 $P_t = (x_0^{(t)}, v^{(t)})$,并计算射击位置 $\text{Pos}_t = x_0^{(t)} + v^{(t)} \times t$。因为超人的真实参数对 $(x_0^*, v^*)$ 必然在序列的某一位 $k$ 出现(即 $P_k = (x_0^*, v^*)$),所以蝙蝠侠最晚在第 $k$ 回合击中目标。



\vspace{1cm}

\section*{5. Brownian Motion }

\subsubsection*{解答}
根据 $q$ 与 $a$ 的大小关系分两种情况讨论:

\textbf{情形 1:$0 < q < a$}

已知起点 $B(0)=0$ 和终点 $B(a)=b$,求中间时刻 $B(q)$ 的条件分布。

根据布朗桥性质,在给定两端点的条件下,中间点的值呈线性插值,并有额外的随机波动:
\begin{align*}
E[B(q) \mid B(a)=b] &= \frac{q}{a} \cdot b \\[0.5em]
\text{Var}(B(q) \mid B(a)=b) &= \frac{q(a-q)}{a}
\end{align*}

\vspace{0.5cm}

\textbf{情形 2:$q > a$}

已知 $B(a)=b$,求未来时刻 $B(q)$ 的条件分布。

利用布朗运动的独立增量性质:$B(q) = B(a) + \underbrace{(B(q)-B(a))}_{\text{独立于} B(a)}$

因为 $B(q)-B(a) \sim \mathcal{N}(0, q-a)$ 且独立于 $B(a)$,所以:
\begin{align*}
E[B(q) \mid B(a)=b] &= b + E[B(q)-B(a)] = b + 0 = b \\[0.5em]
\text{Var}(B(q) \mid B(a)=b) &= \text{Var}(B(q)-B(a)) = q - a
\end{align*}

\newpage

\section*{6. 广播机制 }

\subsubsection*{解答}
\textbf{答案:(2, 3, 3) 选B}

\section*{7. 接雨水 (算法)}

\subsubsection*{解答}
\textbf{算法步骤:}
\begin{enumerate}
    \item 使用动态规划预处理每个位置的\textbf{左侧最高点}和\textbf{右侧最高点}。
    \item 对每个位置,计算能蓄水的高度(木桶短板原理)。
    \item 将蓄水高度乘以对应宽度,累加得到总体积。
\end{enumerate}

\textbf{Python 参考实现:}
\begin{lstlisting}[language=Python]
def trap_rain_weighted(heights, widths):
    """
    计算加权接雨水问题的总体积

    参数:
        heights: 高度数组
        widths: 宽度数组
    返回:
        总蓄水体积
    """
    n = len(heights)
    if n == 0:
        return 0

    # 计算每个位置左侧的最大高度
    left_max = [0] * n
    left_max[0] = heights[0]
    for i in range(1, n):
        left_max[i] = max(left_max[i-1], heights[i])

    # 计算每个位置右侧的最大高度
    right_max = [0] * n
    right_max[n-1] = heights[n-1]
    for i in range(n-2, -1, -1):
        right_max[i] = max(right_max[i+1], heights[i])

    # 计算总蓄水量
    total_water = 0
    for i in range(n):

        water_level = min(left_max[i], right_max[i])
        # 当前位置的蓄水深度
        depth = max(0, water_level - heights[i])
        # 累加体积:深度 × 宽度
        total_water += depth * widths[i]

    return total_water
\end{lstlisting}

\end{document}