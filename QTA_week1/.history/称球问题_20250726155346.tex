\documentclass[12pt, a4paper]{ctexart}

% --- 使用的包 ---
\usepackage{amsmath}   % 提供高级数学公式环境
\usepackage{amssymb}   % 提供数学符号
\usepackage{geometry}  % 用于设置页面边距
\usepackage{ctex}   
\usepackage{amsmath}   % 提供高级数学公式环境,例如 pmatrix
\usepackage{amssymb}   % 提供数学符号
\usepackage{geometry}  % 用于设置页面边距
% --- 页面布局设置 ---
\geometry{a4paper, margin=1in}

% --- 文档标题信息 ---
\title{称球问题}
\author{欧岱松}
\date{\today}

\begin{document}

\maketitle

\section{核心基础}

我们构建单次称重物理结果的数学关系式,来描述每一次的称重结果。
我们定义:
\begin{itemize}
    \item $s_j$: 第 $j$ 号球的真实状态。$s_j=1$ 表示该球\textbf{偏重},$s_j=-1$ 表示该球\textbf{偏轻}。
    \item $W_j$: 在某一次称重中,对第 $j$ 号球的放置方式。$W_j=-1$ 表示放\textbf{左盘},$W_j=1$ 表示放\textbf{右盘},$W_j=0$ 表示\textbf{不放}。
    \item $r$: 本次称重的结果。$r=-1$ 表示\textbf{左盘重},$r=1$ 表示\textbf{右盘重},$r=0$ 表示\textbf{平衡}。
\end{itemize}
通过简单的分析,我们可以得到以下恒成立的公式:
\begin{equation}
    r = W_j \times s_j
\end{equation}

具体来说,
\begin{itemize}
    \item 若球 $j$ \textbf{偏重} ($s_j=1$),放在\textbf{左盘} ($W_j=-1$),则左盘会下沉,结果 $r=-1$。公式:$-1 \times 1 = -1$。 \textbf{吻合}。
    \item 若球 $j$ \textbf{偏轻} ($s_j=-1$),放在\textbf{左盘} ($W_j=-1$),则左盘会上升(右盘下沉),结果 $r=1$。公式:$-1 \times -1 = 1$。 \textbf{吻合}。
    \item 若球 $j$ 放在\textbf{右盘} ($W_j=1$) 且\textbf{偏重} ($s_j=1$),则右盘下沉,结果 $r=1$。公式:$1 \times 1 = 1$。 \textbf{吻合}。
    \item 若球 $j$ \textbf{不放}在天平上 ($W_j=0$),则天平平衡,结果 $r=0$。公式:$0 \times s_j = 0$。 \textbf{吻合}。
\end{itemize}

\section{问题描述}

根据信息论,我们可以通过三次称重来确定次品球。如果第 $j$ 号球是次品,那么上述的核心公式必须对每一次称重都成立。设第 $i$ 次称重的结果为 $r_i$,对球 $j$ 的放置为 $W_{ij}$,则有:
\begin{align*}
    \text{第1次称重: } & r_1 = W_{1j} \times s_j \\
    \text{第2次称重: } & r_2 = W_{2j} \times s_j \\
    \text{第3次称重: } & r_3 = W_{3j} \times s_j
\end{align*}

更进一步,
\begin{itemize}
    \item 我们将三次称重的实际结果组合成一个\textbf{结果向量} $R = (r_1, r_2, r_3)^T$。
    \item 我们将对第 $j$ 号球的三次称重方案组合成一个\textbf{方案向量} $W_j = (W_{1j}, W_{2j}, W_{3j})^T$。这个向量就是我们预先设计的称重矩阵 $W$ 的第 $j$ 列。
\end{itemize}
于是,上面三个的方程可以被合并成一个方程:
\begin{equation}
    \mathbf{R} = s_j \cdot \mathbf{W_j}
\end{equation}

故而原问题被我们转化成如何设计一个方案矩阵,并根据结果来反推出球的状态向量,又因为状态向量为一个全一向量或者0向量,故而最终的次品的方案向量一定是满足
$R = k \times W_j$(并且$k \in \{-1, 1\}$)。
\section{问题求解}

问题求解的本质就是在**求解方程 (2)**。在这个方程中:
\begin{itemize}
    \item 向量 $\mathbf{R}$ 是我们通过实际称重得到的。
    \item 球的编号 $j$ 和它的状态 $s_j$ 是我们想要找出的。
\end{itemize}
由于球的状态 $s_j$ 只有三种可能(1 , -1 或 0 ),故而当方案向量和结果向量不满足$R = k \times W_j$(并且$k \in \{-1, 1\}$)时,其必然不是次品球。接下来我们需要说明,选定特定的方案矩阵,找出与结果向量共线性的向量,这个解存在且唯一。



\section*{一个可行的最优策略矩阵}

满足所有设计原则(唯一列向量、无负向量对、行和为零)的一个经典 $3 \times 12$ 称重矩阵 $W$ 如下。

\subsection*{称重方案描述}
\begin{itemize}
    \item \textbf{第1次称重}: \{1, 2, 3, 4\} vs \{5, 6, 7, 8\}
    \item \textbf{第2次称重}: \{1, 2, 5, 9\} vs \{3, 6, 7, 10\}
    \item \textbf{第3次称重}: \{1, 3, 5, 11\} vs \{4, 6, 9, 12\}
\end{itemize}

\subsection*{对应的称重矩阵 $W$}
下面的矩阵精确地描述了上述方案,其中行代表称重次数,列代表球的编号。
$$
\mathbf{W} =
\begin{pmatrix}
% 球:  1 &  2 &  3 &  4 &  5 &  6 &  7 &  8 &  9 & 10 & 11 & 12 \\
   -1 & -1 & -1 & -1 &  1 &  1 &  1 &  1 &  0 &  0 &  0 &  0 \\ % 称重 1
   -1 & -1 &  1 &  0 & -1 &  1 &  1 &  0 & -1 &  1 &  0 &  0 \\ % 称重 2
   -1 &  0 & -1 &  1 & -1 &  1 &  0 &  0 &  1 &  0 & -1 &  1    % 称重 3
\end{pmatrix}
$$
综上所述,“查找匹配”操作并非一个随意的比对过程,而是求解一个被我们精心设计好的向量方程的直观方法。我们拿着已知的称重结果 $\mathbf{R}$,逐一去检验称重矩阵中的每一列 $\mathbf{W_j}$,看它满足的是情况A还是情况B。

由于我们设计的称重矩阵 $W$ 保证了每一列向量都是独一无二的,且没有一列是另一列的负向量,因此这个求解过程得到的结果必然是**唯一**的。这就从数学上保证了我们能准确无误地找出那个唯一的次品球和它的状态。

\end{document}
