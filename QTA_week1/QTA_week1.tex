\documentclass{article}
\usepackage{amsmath}
\usepackage{amssymb}
\usepackage{geometry}
\usepackage{verbatim}
\usepackage{xcolor}
\usepackage{graphicx}
\usepackage{listings}
\usepackage{ctex}
\usepackage{hyperref} % For clickable links

% Set page margins
\geometry{a4paper, margin=1in}

% Custom commands for problem numbering
\newcounter{problem}
\newcommand{\problem}[1]{\refstepcounter{problem}\subsection*{问题 \arabic{problem}. #1}}

% Styling for answer areas
\newcommand{\answerbox}[1]{\fbox{\parbox{\dimexpr\linewidth-2\fboxsep-2\fboxrule}{#1}}}
\newcommand{\answerlines}[1]{\vspace{#1\baselineskip}\hrulefill}

\begin{document}

\title{QTA2025暑期求职笔试训练营-week1题目}
\author{作答区}
\date{2025年7月21日} % Current date

\maketitle

\section*{题目}

\problem{单败淘汰赛}
64个人进行单败淘汰制竞赛(每一轮随机分组两两PK,决出一位胜者进入下一轮)
\begin{enumerate}
    \item 假如64个人实力有严格顺序,实力强的人一定赢实力弱的人,问最强的人甲、第二强的人乙在决赛相遇的概率是多少?
    \item 假如64个人实力相当,两两相遇胜率均为五五开,问任意两人甲、乙在决赛相遇的概率是多少?
\end{enumerate}


\textbf{答:}对于一个64人的单败淘汰赛,其核心结构如下:
\begin{itemize}
    \item 总共需要进行 $64 - 1 = 63$ 场比赛。
    \item 总共有 $\log_2(64) = 6$ 轮比赛。
    \item 一个选手要赢得冠军,需要连赢6场。要进入决赛,需要连赢5场。
    \item \textbf{随机分组}是关键。我们可以想象,在比赛开始前,64个位置(签位)是固定的,而64个选手被随机地分配到这64个位置上。
\end{itemize}

为了让甲和乙在决赛相遇,一个\textbf{绝对必要的前提条件}是:\textbf{他们必须被分在整个赛区的不同半区}。

整个赛区有64个位置,可以被分成两个各有32个位置的上半区和下半区。

% 注意:您需要有一张名为 bracket_halves.png 的图片在同一个目录下才能显示
% \begin{figure}[h!]
% \centering
% \includegraphics[width=0.6\textwidth]{bracket_halves.png}
% \caption{64人赛程分为两个32人的半区}
% \end{figure}

如果甲和乙在同一个半区,他们必然会在决赛前相遇,其中一人会被淘汰,因此他们不可能在决赛相遇。只有当他们分处不同半区时,才\textbf{有可能}在决赛相遇。

我们可以先计算这个前提条件的概率:
\begin{enumerate}
    \item 想象一下,先把甲随机放在64个位置中的任意一个。
    \item 此时还剩下63个空位。
    \item 甲所在的半区还剩下 $32 - 1 = 31$ 个位置。
    \item 而另一个半区则有32个位置。
    \item 因此,乙被分到另一个半区的概率是:
    $$ P(\text{甲乙在不同半区}) = \frac{\text{另一半区的位置数}}{\text{总剩余位置数}} = \frac{32}{63} $$
\end{enumerate}
现在,我们基于这个共同的前提来分析两种不同情况。

\section{情况一:实力有严格顺序}
\subsection*{问题}
最强的人甲、第二强的人乙在决赛相遇的概率是多少?

\subsection*{建模与计算}
\begin{description}
    \item[步骤1:计算甲乙被分在不同半区的概率。]
    如上所述,这个概率是 $\frac{32}{63}$。
    
    \item[步骤2:分析在该前提下,两人能否进入决赛。]
    \begin{itemize}
        \item \textbf{甲的路径}:甲是所有64人中实力最强的。一旦他被分入一个半区,无论对手是谁,他都必然会赢。因此,只要甲乙不在同一个半区,甲就\textbf{一定}会赢得他所在半区的所有比赛,进入决赛。这个概率是1。
        \item \textbf{乙的路径}:乙是第二强的选手。只要最强的甲不在他所在的半区(这正是我们的前提条件),那么乙就是他所在半区里实力最强的选手。因此,他也\textbf{一定}会赢得他所在半区的所有比赛,进入决赛。这个概率也是1。
    \end{itemize}
    
    \item[步骤3:合并概率。]
    甲乙在决赛相遇的概率 = (甲乙在不同半区的概率) × (在该前提下甲进入决赛的概率) × (在该前提下乙进入决赛的概率)。
    \begin{align*}
        P(\text{甲乙决赛相遇}) &= P(\text{甲乙在不同半区}) \times P(\text{甲进决赛} | \text{不同半区}) \times P(\text{乙进决赛} | \text{不同半区}) \\
        &= \frac{32}{63} \times 1 \times 1 = \frac{32}{63}
    \end{align*}
\end{description}

\subsection*{结论}
假如实力强的人一定赢,那么最强者甲和次强者乙在决赛相遇的概率是 $\mathbf{\frac{32}{63}}$,约等于 $50.79\%$。

\section{情况二:实力相当(胜率五五开)}
\subsection*{问题}
任意两人甲、乙在决赛相遇的概率是多少?

\subsection*{建模与计算}
\begin{description}
    \item[步骤1:计算甲乙被分在不同半区的概率。]
    这与第一种情况完全相同。为了让两人能在决赛相遇,他们必须从不同的半区开始。这个概率依然是 $\frac{32}{63}$。
    
    \item[步骤2:分析在该前提下,两人各自进入决赛的概率。]
    \begin{itemize}
        \item \textbf{甲的路径}:要进入决赛,甲需要在他所在的半区(32人)中连续赢得5场比赛($2^5 = 32$)。由于所有人的实力都相当,每一场比赛的胜率都是 $\frac{1}{2}$。甲连赢5场的概率是:
        $$ P(\text{甲进决赛}) = \left(\frac{1}{2}\right)^5 = \frac{1}{32} $$
        \item \textbf{乙的路径}:同样地,乙要进入决赛,也必须在他所在的半区(另外32人)中连续赢得5场比赛。乙连赢5场的概率是:
        $$ P(\text{乙进决赛}) = \left(\frac{1}{2}\right)^5 = \frac{1}{32} $$
        由于甲和乙在不同半区,他们各自的晋级之路是相互独立的事件。
    \end{itemize}
    
    \item[步骤3:合并概率。]
    甲乙在决赛相遇的概率 = (甲乙在不同半区的概率) × (甲在他所在半区获胜的概率) × (乙在他所在半区获胜的概率)。
    \begin{align*}
        P(\text{甲乙决赛相遇}) &= P(\text{甲乙在不同半区}) \times P(\text{甲连赢5场}) \times P(\text{乙连赢5场}) \\
        &= \frac{32}{63} \times \frac{1}{32} \times \frac{1}{32} = \frac{1}{63 \times 32} = \frac{1}{2016}
    \end{align*}
\end{description}

\subsection*{另一种建模思路(结果相同)}
\begin{enumerate}
    \item \textbf{甲进入决赛的概率是多少?} \\
    甲需要连赢5场比赛,概率是 $(\frac{1}{2})^5 = \frac{1}{32}$。
    \item \textbf{在甲已经进入决赛的前提下,乙成为另一名决赛选手的概率是多少?} \\
    甲已经锁定了决赛的一个席位。剩下63名选手将争夺另一个席位。因为所有人的实力都一样,所以这63人中任何一人进入决赛的概率是均等的。因此,乙成为另一名决赛选手的概率是 $\frac{1}{63}$。
    \item \textbf{合并概率。}
    \begin{align*}
     P(\text{甲乙决赛相遇}) &= P(\text{甲进决赛}) \times P(\text{乙是另一位决赛选手} | \text{甲已进决赛}) \\
     &= \frac{1}{32} \times \frac{1}{63} = \frac{1}{2016}
    \end{align*}
\end{enumerate}


\subsection*{结论}
假如所有人实力相当,任意两人甲、乙在决赛相遇的概率是 $\mathbf{\frac{1}{2016}}$,约等于 $0.0496\%$。

\problem{离散停问题}
一直生成$(0,1)$上独立同分布的随机数,直到新的数比上一个小就停止生成,问生成随机数个数的期望?


依据问题,我们考虑一个离散时间随机过程 $\{X_n\}_{n \ge 1}$,其中 $X_n$ 是一系列独立同分布(i.i.d.)的随机变量,服从于状态空间 $S = (0,1)$ 上的均匀分布,即 $X_n \sim U(0,1)$。定义 $\mathcal{F}_n = \sigma(X_1, X_2, \dots, X_n)$ 为该过程的自然信息流。$\mathcal{F}_n$ 代表了截至时间 $n$ 的所有历史信息。

我们感兴趣的随机变量 $N$(生成的数的个数)是一个关于信息流 $\{\mathcal{F}_n\}$ 的停止时。其定义为:
    $$ N := \inf \{n \ge 2 : X_n < X_{n-1}\} $$


我们需要求解期望停止时生成数的个数的期望$\mathbb{E}[N]$,当生成数为正整数时,其期望可以表示为:
$$ \mathbb{E}[N] = \sum_{k=1}^{\infty} P(N \ge k) $$
这里的 $P(N \ge k)$ 是过程在时间 $k-1$ 时的存活概率。

过程存活至时间 $k$(即 $N \ge k$)的条件是,在所有 $j=2, \dots, k-1$ 时刻均未停止。这等价于前 $k-1$ 个随机变量构成了一个严格递增的序列。
$$ \{N \ge k\} \iff \{X_1 < X_2 < \dots < X_{k-1}\} $$
该事件是关于过程样本路径 $(X_1, \dots, X_{k-1})$ 的一个性质。由于 $\{X_n\}$ 是独立同分布的,其联合分布具有排列对称性,任意一种特定排序的概率均为 $\frac{1}{(k-1)!}$。
$$ P(N \ge k) = P(X_1 < X_2 < \dots < X_{k-1}) = \frac{1}{(k-1)!} $$
这个公式对所有 $k \ge 1$ 均成立,严格的数学计算见附录

因此,停止时的期望为:
\begin{align*}
\mathbb{E}[N] &= \sum_{k=1}^{\infty} P(N \ge k) \\
&= \sum_{k=1}^{\infty} \frac{1}{(k-1)!} \\
&= \frac{1}{0!} + \frac{1}{1!} + \frac{1}{2!} + \frac{1}{3!} + \dots
\end{align*}
令 $j=k-1$,上式变为:
$$ \mathbb{E}[N] = \sum_{j=0}^{\infty} \frac{1}{j!} $$
这个级数是自然常数 $e$ 的泰勒展开式。

\problem{次品称重问题}
你有12个铁球,其中包含一个次品球是比其它轻或者比其它重的,给你一个天平,最少称重几次可以保证找出这个球?

\answerbox{\answerlines{5}}

\problem{领导的红包}
时近春节,部门领导要给下属发年终奖。你的领导拿着三个红包A、B、C,其中只有一个红包是200元的,其余两个都是50元,让你从中挑选一个红包,但是不能打开,假设你选择了A红包,然后领导当着你的面拆开了B红包,发现里面有50元,之后领导问你是否要用手中的A红包换剩下的C红包?假设领导知道每个红包内的钱数且一定会打开50元的红包。请给出你的思路和理由。

\answerbox{\answerlines{8}}

\problem{幽灵宝藏}
10个房间排成一条直线,只有一个房间有宝藏。每天宝藏都会移动到某个相邻房间。每天只能检查一个房间,请问至少需要多少天可以确保找到宝藏?给出你的策略。

\answerbox{\answerlines{8}}

\problem{归并排序的复杂度}
侯哥在一台电脑上编写了一个程序对256个字符串进行排序(采用MergeSort归并排序)用时2秒钟,那么用这个程序在给512个字符串进行排序的期望时间是多少秒?

\answerbox{\answerlines{5}}

\problem{Python小坑}
python语法中的一些细节规则是量化OA选择题中经常出现的考察点~
\begin{enumerate}
    \item python的copy函数

    下述代码的输出结果是什么?
\begin{verbatim}
import copy
l=[1,2,3]
x={'a':l}
y= copy.copy (x)
y['a'][0]=11
print(x['a'][0])
\end{verbatim}
输出结果为:[11]
\\
\textbf{原因:这是因为copy.copy方法执行的是浅拷贝,其创建了一个新的字典,但字典内部的元素是直接复制原容器(x)的引用,故而,x,y是两个不同的对象,
但是其中的值是相同的,故而对y的值进行新的赋值,y的值也会发生改变。
如果要使得对y的赋值不改变x的值,应该使用深拷贝,y = copy.deepcopy(x)。}


    \item python函数中的默认参数值

    下述代码会输出什么结果?
\begin{verbatim}
def fast(items=[]):
    items.append(1)
    return items

print(fast())
print(fast())
\end{verbatim}
    A. [1,1] [1,1]
    B. [1] [1]
    C. [1,1] [1]
    D. [1] [1,1]
\end{enumerate}

输出结果为:D 
\\\textbf{原因:在Python中,函数也是对象,当第一次调用时,函数fast执行时,Python会在内存中创建一个空的列表对象 [],并将这个唯一的列表对象的引用,
在Python中,函数也是对象,当第一次调用时,函数fast执行时,Python会在内存中创建一个空的列表对象 [],并将这个唯一的列表对象的引用,
而在函数定义时创建后,就会一直存D。直到程序结束。第一次调用时,item首先创建一个空列表,函数fast执行时,items添加了元素,再次调用时,items列表对象
的默认值变成[1],执行函数后,变成 [1,1],故而选D}


\section{附录}
\subsection{均匀分布推导}
我们旨在计算概率 $P(N \ge k) = P(X_1 < X_2 < \dots < X_{k-1})$,其中 $X_i \sim U(0,1)$ 且相互独立。该概率等价于计算下面这个 $(k-1)$ 维多重积分:
$$ P(N \ge k) = \int_0^1 \int_0^{x_{k-1}} \int_0^{x_{k-2}} \cdots \int_0^{x_2} 1 \, dx_1 \, dx_2 \cdots dx_{k-2} \, dx_{k-1} $$
其具体的推导过程如下:
\begin{align*}
    P(N \ge k) &= \int_0^1 \int_0^{x_{k-1}} \cdots \int_0^{x_3} \int_0^{x_2} 1 \, dx_1 \, dx_2 \cdots dx_{k-1} \\
    \\
    &= \int_0^1 \int_0^{x_{k-1}} \cdots \int_0^{x_3} \left( [x_1]_0^{x_2} \right) \, dx_2 \cdots dx_{k-1} \quad &&\text{第一步:对 $x_1$ 积分} \\
    &= \int_0^1 \int_0^{x_{k-1}} \cdots \int_0^{x_3} x_2 \, dx_2 \cdots dx_{k-1} \\
    \\
    &= \int_0^1 \int_0^{x_{k-1}} \cdots \int_0^{x_4} \left( \left[ \frac{x_2^2}{2} \right]_0^{x_3} \right) \, dx_3 \cdots dx_{k-1} \quad &&\text{第二步:对 $x_2$ 积分} \\
    &= \int_0^1 \int_0^{x_{k-1}} \cdots \int_0^{x_4} \frac{x_3^2}{2!} \, dx_3 \cdots dx_{k-1} \\
    \\
    &\qquad\qquad \vdots \quad &&\text{重复此过程} \\
    \\
    &= \int_0^1 \frac{x_{k-1}^{k-2}}{(k-2)!} \, dx_{k-1} \quad &&\text{在对 $x_{k-2}$ 积分后,得到的结果}\\
    \\
    &= \frac{1}{(k-2)!} \left[ \frac{x_{k-1}^{k-1}}{k-1} \right]_0^1 \quad &&\text{最后一步:对 $x_{k-1}$ 积分} \\
    \\
    &= \frac{1}{(k-2)!} \left( \frac{1^{k-1}}{k-1} - \frac{0^{k-1}}{k-1} \right) \\
    \\
    &= \frac{1}{(k-2)!} \cdot \frac{1}{k-1} \\
    \\
    &= \frac{1}{(k-1)!}
    \end{align*}
\end{document}