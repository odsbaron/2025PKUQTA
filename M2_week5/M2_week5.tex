\documentclass[11pt, a4paper]{article}
\usepackage[UTF8]{ctex}
\usepackage{amsmath, amssymb, amsthm}
\usepackage{geometry}
\usepackage{listings}
\usepackage{xcolor}

\geometry{left=2.5cm, right=2.5cm, top=2.5cm, bottom=2.5cm}

\definecolor{codegreen}{rgb}{0,0.6,0}
\definecolor{codegray}{rgb}{0.5,0.5,0.5}
\definecolor{codepurple}{rgb}{0.58,0,0.82}
\definecolor{backcolour}{rgb}{0.95,0.95,0.92}

\lstdefinestyle{mystyle}{
    backgroundcolor=\color{backcolour},
    commentstyle=\color{codegreen},
    keywordstyle=\color{magenta},
    numberstyle=\tiny\color{codegray},
    stringstyle=\color{codepurple},
    basicstyle=\ttfamily\footnotesize,
    breakatwhitespace=false,
    breaklines=true,
    captionpos=b,
    keepspaces=true,
    numbers=left,
    numbersep=5pt,
    showspaces=false,
    showstringspaces=false,
    showtabs=false,
    tabsize=2
}
\lstset{style=mystyle}

\title{\textbf{QTA 笔面试刷题 Week 5 解析}}
\author{QTA}
\date{December 2025}

\begin{document}

\maketitle
\tableofcontents
\newpage

\section{相关系数为 $\rho$ 的取值范围}

\subsection{题目描述}
设随机变量 $X, Y, Z$ 两两相关系数均为 $\rho$,其相关矩阵
$$
R = \begin{pmatrix}
1 & \rho & \rho \\
\rho & 1 & \rho \\
\rho & \rho & 1
\end{pmatrix}
$$
要求矩阵半正定,求 $\rho$ 的范围。

\subsection{解答与推导}
$R$ 可写成 $R = (1-\rho)I + \rho J$,全 1 阵在三维的特征值是 $3$(重数 1)和 $0$(重数 2),因此 $R$ 的特征值直接为 $1+2\rho,\,1-\rho,\,1-\rho$。半正定只需这些特征值非负,$1+2\rho \ge 0,\,1-\rho \ge 0$,解得 $-\tfrac{1}{2} \le \rho \le 1$。

\textbf{结论:} $\rho \in [-\tfrac{1}{2}, 1]$。

\section{连续 6 个 6 的期望投掷次数}

\subsection{题目描述}
公平骰子掷到连续 6 个 6 的期望投掷次数。

\subsection{解答与推导}
设 $E_n$ 为达成连续 $n$ 个 6 的期望掷数。已连到 $n-1$ 个 6 时再掷一次,先付出 1 次尝试;若掷出 6(概率 $1/6$)则直接完成,若不是 6(概率 $5/6$)连击清零,又要重新经历期望为 $E_n$ 的过程。全期望给出
$$E_n = E_{n-1} + 1 + \frac{5}{6}E_n \quad\Rightarrow\quad E_n = 6E_{n-1} + 6,$$
且 $E_0=0$。递推可得 $E_n=\sum_{i=1}^n 6^i$,于是
$$E_6 = \frac{6(1-6^6)}{1-6} = 55986.$$

\textbf{结论:} 期望掷数为 55986。

\section{随机落座(疯子坐飞机)}

\subsection{题目描述}
第 1 人随机就座,其余人如果自己的座位空则入座,否则随机。求第 $N$ 人坐对的概率。

\subsection{解答与推导}
混乱只在第 1 人被迫随机时出现,此后每当有人随机就座,空位集合只会在“1 号座位”和“$N$ 号座位”之间传递,直到其中之一被选中才结束:若先被选中的是 1 号,则后续人人归位,$N$ 人坐对;若先被选中的是 $N$ 号,则 $N$ 人坐错;落在其他编号只是在把同样的局面递给下一个人。1 号与 $N$ 号被率先选中的概率全程对称,因此各为 $\tfrac12$。

\textbf{结论:} 第 $N$ 个人坐对的概率为 $0.5$。

\section{外星生物颜色}

\subsection{题目描述}
初始红 20、绿 21、蓝 22。规则:两种不同颜色变为两个第三种颜色。问能否最终全为同一种颜色。

\subsection{解答与推导}
设三色数量为 $n_1,n_2,n_3$。一次操作让其中两色各减 1、第三色加 2,模 3 等价于所有颜色数同时加 2,因此任意两色数量差在模 3 下保持不变。初始
$$n_{\text{红}}\equiv2,\quad n_{\text{绿}}\equiv0,\quad n_{\text{蓝}}\equiv1 \pmod 3,$$
两两之差模 3 分别为 $2,1,1$,都不为 0。若最终全为同色则任意差为 0(模 3 也为 0),与不变量冲突,故无法实现。

\textbf{结论:} 不可能变成同一种颜色。

\section{比较排序的时间复杂度下界}

\subsection{题目描述}
证明任意基于比较的排序算法,最坏情况时间复杂度下界为 $O(n\log n)$。

\subsection{证明}
用决策树刻画比较排序,对 $n$ 个元素共有 $n!$ 种排列,必须对应 $n!$ 个叶子。二叉决策树高度为 $h$ 时叶子最多 $2^h$ 个,因此需 $2^h\ge n!$,得 $h\ge\log_2(n!)$。由斯特林公式 $\ln(n!)\approx n\ln n-n$ 可知 $\log_2(n!)$ 量级为 $n\log n$,故最坏情况比较次数下界为 $\Omega(n\log n)$。

\textbf{结论:} 最坏时间下界 $\Omega(n\log n)$。

\section{交易逆序对总数}

\subsection{题目描述}
给定股票价格序列,若前一天价格高于后一天则构成逆序对。$N\le 50000$,求逆序对总数。

\subsection{解答与算法}
暴力为 $O(N^2)$,需用 $O(N\log N)$。在归并排序的合并阶段,左右子数组有序;当遇到左侧元素 $L[i]>R[j]$,则 $L[i]$ 以及其后的 $(\text{mid}-i+1)$ 个元素都大于 $R[j]$,一次性贡献同样数量的逆序对,然后写入 $R[j]$ 继续归并。累积计数即可得到总数。

\subsection{代码实现 (Python)}

\begin{lstlisting}[language=Python]
class Solution:
    def reversePairs(self, record: list[int]) -> int:
        self.count = 0
        self.merge_sort(record, 0, len(record) - 1)
        return self.count

    def merge_sort(self, nums, left, right):
        if left >= right:
            return
        mid = (left + right) // 2
        self.merge_sort(nums, left, mid)
        self.merge_sort(nums, mid + 1, right)
        self.merge(nums, left, mid, right)

    def merge(self, nums, left, mid, right):
        temp = []
        i, j = left, mid + 1
        while i <= mid and j <= right:
            if nums[i] <= nums[j]:
                temp.append(nums[i]); i += 1
            else:
                self.count += (mid - i + 1)
                temp.append(nums[j]); j += 1
        while i <= mid:
            temp.append(nums[i]); i += 1
        while j <= right:
            temp.append(nums[j]); j += 1
        for k, v in enumerate(temp):
            nums[left + k] = v
\end{lstlisting}

\end{document}
