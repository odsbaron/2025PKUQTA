\documentclass[11pt, a4paper]{article}
\usepackage[UTF8]{ctex}
\usepackage{amsmath, amssymb, amsthm}
\usepackage{geometry}
\usepackage{listings}
\usepackage{xcolor}
\usepackage{fancyhdr}

\geometry{left=2.5cm, right=2.5cm, top=2.5cm, bottom=2.5cm}

\definecolor{codegreen}{rgb}{0,0.6,0}
\definecolor{codegray}{rgb}{0.5,0.5,0.5}
\definecolor{codepurple}{rgb}{0.58,0,0.82}
\definecolor{backcolour}{rgb}{0.95,0.95,0.92}

\lstdefinestyle{mystyle}{
    backgroundcolor=\color{backcolour},
    commentstyle=\color{codegreen},
    keywordstyle=\color{magenta},
    numberstyle=\tiny\color{codegray},
    stringstyle=\color{codepurple},
    basicstyle=\ttfamily\footnotesize,
    breakatwhitespace=false,
    breaklines=true,
    captionpos=b,
    keepspaces=true,
    numbers=left,
    numbersep=5pt,
    showspaces=false,
    showstringspaces=false,
    showtabs=false,
    tabsize=2
}
\lstset{style=mystyle}

\title{\textbf{QTA 笔面试刷题 Week 5 完整解析}}
\author{QTA}
\date{December 2025}

\begin{document}

\maketitle
\tableofcontents
\newpage

\section{相关系数为 $\rho$ 的取值范围}

\subsection{题目描述}
设随机变量 $X, Y, Z$ 两两相关系数均为 $\rho$,其相关矩阵为
$$
R = \begin{pmatrix}
1 & \rho & \rho \\
\rho & 1 & \rho \\
\rho & \rho & 1
\end{pmatrix}
$$
要求联合随机变量的相关矩阵半正定(即 $R \succeq 0$),求 $\rho$ 的取值范围。

\subsection{解答与推导}
矩阵 $R$ 可写为 $R = (1-\rho)I + \rho J$,其中 $I$ 是 $3 \times 3$ 的单位矩阵,$J$ 是全 1 矩阵。全 1 矩阵 $J_{3 \times 3}$ 的特征值是 $3$(重数 1,对应特征向量 $[1,1,1]^T$)和 $0$(重数 2),代入线性组合得到
\begin{align*}
    \lambda_1 &= (1-\rho) + 3\rho = 1 + 2\rho, \\
    \lambda_2 &= (1-\rho) + 0 = 1 - \rho, \\
    \lambda_3 &= (1-\rho) + 0 = 1 - \rho\text{。}
\end{align*}
半正定要求所有特征值非负,因而 $\lambda_i \ge 0$ 给出
\begin{cases}
    1 + 2\rho \ge 0 \implies \rho \ge -\frac{1}{2}, \\
    1 - \rho \ge 0 \implies \rho \le 1\text{。}
\end{cases}

\textbf{结论:} $\rho$ 的取值范围是 $[-\frac{1}{2}, 1]$。

\section{掷出连续 6 个 6 的期望投掷次数}

\subsection{题目描述}
一个均匀的骰子,掷出连续 6 个 6 的期望次数是多少?

\subsection{解答与推导}
记 $E_n$ 为掷出连续 $n$ 个 6 所需的期望次数。把状态视为“已连中 $n-1$ 个 6”,此时再掷一次必然增加 1 次尝试:若结果是 6(概率 $1/6$)直接达成连续 $n$ 个 6;若不是 6(概率 $5/6$)连击被清零,又回到重新追求 $E_n$ 的过程。全期望公式写为
$$ E_n = E_{n-1} + 1 + \frac{5}{6}E_n \quad\Rightarrow\quad E_n = 6E_{n-1} + 6. $$
已知 $E_0 = 0$,得到
\begin{align*}
    E_1 &= 6,\\
    E_2 &= 6^2 + 6,\\
    E_3 &= 6^3 + 6^2 + 6,\\
    &\dots\\
    E_6 &= \sum_{i=1}^{6} 6^i = \frac{6(1-6^6)}{1-6} = 55986\text{。}
\end{align*}

\textbf{结论:} 期望需要 55986 次。

\section{随机落座(疯子坐飞机)}

\subsection{题目描述}
$N$ 个座位,第一人(“疯子”)随机就座。之后正常人若座位空着就坐,否则随机就座。求第 $N$ 个人坐到自己座位的概率。

\subsection{解答与推导}
除去第 1 人和第 $N$ 人,其他人只有在“被迫随机”时才会打乱秩序,而这种随机会把空位集继续留给后面的人。混乱只可能在 1 号座位或 $N$ 号座位终止:若某次随机落在 1 号座位,后续人人归位,最终第 $N$ 个人必然坐对;若落在 $N$ 号座位,则第 $N$ 个人必然坐错;落在其他编号则把同样的局面递给下一人。因为 1 号与 $N$ 号在整个过程中被随机选中的机会始终对称,最终停在 1 号与停在 $N$ 号的概率相等,各为 $\frac{1}{2}$。

\textbf{结论:} 第 $N$ 个人坐对的概率为 $0.5$。

\section{外星生物颜色}

\subsection{题目描述}
初始红 20、绿 21、蓝 22。规则:两种不同颜色 $\to$ 生成两个第三种颜色。问最终能否全部变成同一种颜色?

\subsection{解答与推导}
设三色数量为 $n_1, n_2, n_3$。一次操作使两种颜色各减 1、第三种加 2,对应增量向量 $(-1,-1,+2)$ 及其排列,模 3 都等价于为每个颜色数加 2。于是模 3 下任意两色的差保持不变:
\begin{align*}
    (n_1-1)-(n_2-1)&=n_1-n_2,\\
    (n_1-1)-(n_3+2)&=n_1-n_3-3\equiv n_1-n_3 \pmod 3\text{,其余同理。}
\end{align*}
初态有 $n_{\text{红}}=20\equiv 2,\,n_{\text{绿}}=21\equiv 0,\,n_{\text{蓝}}=22\equiv 1$,两两差模 3 分别为 $2,1,1$,均非 0。若最终全为同色,则任意两色数量差应为 0(模 3 亦为 0),与不变量矛盾,因而无法达成。

\textbf{结论:} 不可能全部变成同一种颜色。

\section{比较排序的时间复杂度}

\subsection{题目描述}
证明任何基于比较的排序算法,最坏情况时间复杂度下界为 $O(n \log n)$。

\subsection{证明}
用决策树模型描述比较排序:对 $n$ 个待排序元素,所有排列共有 $n!$ 种,算法需要通过比较把输入引到对应的叶子节点。每次比较把可能性空间一分为二,于是决策树是一棵二叉树,最坏情况下的比较次数对应树高 $h$,而拥有 $2^h$ 个叶子才能覆盖全部 $n!$ 个排列,所以 $2^h \ge n!$。取对数得 $h \ge \log_2(n!)$,利用斯特林公式 $\ln(n!) \approx n\ln n - n$,可知 $\log_2(n!)$ 量级为 $n \log_2 n$,从而得出最坏情况下界 $\Omega(n \log n)$。

\textbf{结论:} 任意比较排序的最坏时间复杂度下界为 $\Omega(n \log n)$。

\section{交易逆序对总数}

\subsection{题目描述}
输入股票记录,若前一天股价高于后一天股价,则构成“交易逆序对”。数据范围 $N \le 50000$,求逆序对总数。

\subsection{解答与算法}
暴力双层循环是 $O(N^2)$,需改用 $O(N \log N)$。在归并排序的 `merge` 阶段,左右子数组已各自有序:当遇到左侧元素 $L[i] > R[j]$ 时,$L[i]$ 及其之后的 $(\text{mid}-i+1)$ 个元素都大于 $R[j]$,一次性贡献同样数量的逆序对,再把 $R[j]$ 写入临时数组继续归并。这样在合并过程中顺带累加逆序对即可。

\subsection{代码实现 (Python)}

\begin{lstlisting}[language=Python]
class Solution:
    def reversePairs(self, record: list[int]) -> int:
        self.count = 0
        self.merge_sort(record, 0, len(record) - 1)
        return self.count

    def merge_sort(self, nums, left, right):
        if left >= right:
            return
        
        mid = (left + right) // 2
        self.merge_sort(nums, left, mid)
        self.merge_sort(nums, mid + 1, right)
        self.merge(nums, left, mid, right)

    def merge(self, nums, left, mid, right):
        temp = []
        i, j = left, mid + 1
        
        while i <= mid and j <= right:
            if nums[i] <= nums[j]:
                temp.append(nums[i])
                i += 1
            else:
                # 逆序对:左侧剩余元素都大于 nums[j]
                self.count += (mid - i + 1)
                temp.append(nums[j])
                j += 1
        
        while i <= mid:
            temp.append(nums[i])
            i += 1
        while j <= right:
            temp.append(nums[j])
            j += 1
            
        for k in range(len(temp)):
            nums[left + k] = temp[k]
\end{lstlisting}

\end{document}
