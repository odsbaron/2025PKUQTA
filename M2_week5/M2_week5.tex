\documentclass[11pt, a4paper]{article}
\usepackage[UTF8]{ctex} % 支持中文
\usepackage{amsmath, amssymb, amsthm} % 数学公式库
\usepackage{geometry} % 页面设置
\usepackage{listings} % 代码块支持
\usepackage{xcolor} % 颜色支持
\usepackage{fancyhdr} % 页眉页脚

% 页面边距设置
\geometry{left=2.5cm, right=2.5cm, top=2.5cm, bottom=2.5cm}

% 代码块样式设置
\definecolor{codegreen}{rgb}{0,0.6,0}
\definecolor{codegray}{rgb}{0.5,0.5,0.5}
\definecolor{codepurple}{rgb}{0.58,0,0.82}
\definecolor{backcolour}{rgb}{0.95,0.95,0.92}

\lstdefinestyle{mystyle}{
    backgroundcolor=\color{backcolour},   
    commentstyle=\color{codegreen},
    keywordstyle=\color{magenta},
    numberstyle=\tiny\color{codegray},
    stringstyle=\color{codepurple},
    basicstyle=\ttfamily\footnotesize,
    breakatwhitespace=false,         
    breaklines=true,                 
    captionpos=b,                    
    keepspaces=true,                 
    numbers=left,                    
    numbersep=5pt,                  
    showspaces=false,                
    showstringspaces=false,
    showtabs=false,                  
    tabsize=2
}
\lstset{style=mystyle}

% 标题信息
\title{\textbf{QTA 笔面试刷题 Week 5 完整解析}}
\author{QTA}
\date{December 2025}

\begin{document}

\maketitle
\tableofcontents
\newpage

\section{相关系数为 $\rho$ 的取值范围}

\subsection{题目描述}
设随机变量 $X, Y, Z$ 两两相关系数均为 $\rho$,则其相关矩阵为:
$$
R = \begin{pmatrix} 
1 & \rho & \rho \\ 
\rho & 1 & \rho \\ 
\rho & \rho & 1 
\end{pmatrix}
$$
要求联合随机变量的相关矩阵为半正定(即 $R \succeq 0$),求 $\rho$ 的取值范围。

\subsection{解答与推导}
矩阵 $R$ 可以写成 $R = (1-\rho)I + \rho J$,其中 $I$ 是 $3 \times 3$ 的单位矩阵,$J$ 是全 1 矩阵。

\begin{enumerate}
    \item \textbf{求解特征值:}
    全 1 矩阵 $J_{3 \times 3}$ 的特征值为 $3$(重数为1,对应特征向量 $[1,1,1]^T$)和 $0$(重数为2)。
    因此,矩阵 $R$ 的特征值 $\lambda$ 为:
    \begin{align*}
        \lambda_1 &= (1-\rho) \cdot 1 + \rho \cdot 3 = 1 + 2\rho \\
        \lambda_2 &= (1-\rho) \cdot 1 + \rho \cdot 0 = 1 - \rho \\
        \lambda_3 &= (1-\rho) \cdot 1 + \rho \cdot 0 = 1 - \rho
    \end{align*}

    \item \textbf{半正定条件:}
    矩阵半正定要求所有特征值非负,即 $\lambda_i \ge 0$:
    \begin{cases}
        1 + 2\rho \ge 0 \implies \rho \ge -\frac{1}{2} \\
        1 - \rho \ge 0 \implies \rho \le 1
    \end{cases}
\end{enumerate}

\textbf{结论:} $\rho$ 的取值范围是 $[-\frac{1}{2}, 1]$。

\section{掷出连续 6 个 6 的期望投掷次数}

\subsection{题目描述}
一个均匀的骰子,掷出连续 6 个 6 的期望次数是多少?

\subsection{解答与推导}
设 $E_n$ 为掷出连续 $n$ 个 6 所需的期望次数。
我们可以建立递推关系:
\begin{itemize}
    \item 假设我们已经掷出了连续 $n-1$ 个 6(耗费期望 $E_{n-1}$ 次)。
    \item 再掷一次(第 $E_{n-1} + 1$ 次):
    \begin{itemize}
        \item 如果是 6(概率 $1/6$),则达成连续 $n$ 个 6,结束。
        \item 如果不是 6(概率 $5/6$),则连续中断,一切从头开始(需要重新掷 $E_n$ 次)。
    \end{itemize}
\end{itemize}

根据全期望公式:
$$ E_n = E_{n-1} + 1 + \frac{1}{6}(0) + \frac{5}{6}(E_n) $$
化简得:
$$ \frac{1}{6}E_n = E_{n-1} + 1 \implies E_n = 6E_{n-1} + 6 $$

已知 $E_0 = 0$,则:
\begin{align*}
    E_1 &= 6 \\
    E_2 &= 6(6) + 6 = 6^2 + 6 \\
    E_3 &= 6(6^2 + 6) + 6 = 6^3 + 6^2 + 6 \\
    &\dots \\
    E_6 &= \sum_{i=1}^{6} 6^i
\end{align*}

计算等比数列求和:
$$ E_6 = \frac{6(1-6^6)}{1-6} = \frac{6(46656-1)}{5} = 55986 $$

\textbf{结论:} 期望次数为 55986 次。

\section{随机落座(疯子坐飞机)}

\subsection{题目描述}
$N$ 个座位,第 1 人(精神病)随机坐。后续正常人若座位空着就坐,被占则随机坐。求第 $N$ 个人坐到自己座位的概率。

\subsection{解答与推导}
考虑第 1 个座位和第 $N$ 个座位的对称性。对于任何一个人(从第 1 人到第 $N-1$ 人),如果他需要随机选择座位:
\begin{itemize}
    \item 若选中 **1号座位**:由于1号座位对应的是那个“导致混乱的源头”(第1个人本该坐的位置),一旦1号座位被填满,后续所有人都可以坐回自己的位置,包括第 $N$ 个人。
    \item 若选中 **$N$号座位**:第 $N$ 个人的位置被占,第 $N$ 个人必输。
    \item 若选中 **其他座位**($2$ 到 $N-1$):混乱继续传递给下一个人。
\end{itemize}

直到游戏结束前,**1号座位**和**$N$号座位**在空座位集合中被选中的概率始终是相等的。因此,在这两个座位中,1号座位先被选中的概率等于$N$号座位先被选中的概率。

$$ P(\text{第N人坐对}) = P(\text{1号座位先于N号座位被选中}) = \frac{1}{2} $$

\textbf{结论:} 概率为 $0.5$。

\section{外星生物颜色}

\subsection{题目描述}
红20,绿21,蓝22。规则:2个不同色 $\to$ 2个第三种色。问是否可能最后全为一种颜色?

\subsection{解答与推导}
设三种颜色的数量分别为 $n_1, n_2, n_3$。每次变换(例如红+绿$\to$蓝),数量变化向量为 $(-1, -1, +2)$。
考察模 3 的不变量。注意每次变化中:
$$ \Delta n_i \equiv -1 \equiv 2 \pmod 3 \quad \text{或} \quad \Delta n_i = +2 \equiv 2 \pmod 3 $$
这意味着在模 3 意义下,所有颜色的数量都在同时增加 2。
更关键的不变量是**任意两种颜色的数量之差**(模 3):
\begin{align*}
    (n_1 - 1) - (n_2 - 1) &= n_1 - n_2 \\
    (n_1 - 1) - (n_3 + 2) &= n_1 - n_3 - 3 \equiv n_1 - n_3 \pmod 3
\end{align*}
即:**任意两种颜色的数量差在模 3 下保持不变。**

初始状态:
$$ n_{\text{红}}=20 \equiv 2, \quad n_{\text{绿}}=21 \equiv 0, \quad n_{\text{蓝}}=22 \equiv 1 $$
两两之差模 3 分别为 $2, 1, 1$(均不为 0)。

目标状态(假设全变成蓝色):
$$ n_{\text{红}}=0, \quad n_{\text{绿}}=0, \quad n_{\text{蓝}}=63 $$
此时 $n_{\text{红}} - n_{\text{绿}} = 0 \equiv 0 \pmod 3$。

由于初始状态 $n_{\text{红}} - n_{\text{绿}} = -1 \equiv 2 \neq 0$,故无法达到目标状态。

\textbf{结论:} 不可能。

\section{比较排序的时间复杂度}

\subsection{题目描述}
证明任何基于比较的排序算法,其最坏情况下的时间复杂度下界为 $O(n \log n)$。

\subsection{证明}
利用决策树模型(Decision Tree Model):
\begin{enumerate}
    \item 对于 $n$ 个待排序元素,其全排列共有 $n!$ 种情况,这些是排序算法必须区分的输出状态(叶子节点)。
    \item 比较排序的每一步操作(比较 $a < b$)将可能性空间一分为二。这就构成了一棵二叉树。
    \item 设树的高度为 $h$(对应最坏情况下的比较次数)。二叉树最多有 $2^h$ 个叶子节点。
    \item 为了覆盖所有可能的排列,必须满足:
    $$ 2^h \ge n! $$
    \item 两边取对数:
    $$ h \ge \log_2(n!) $$
    \item 根据斯特林公式(Stirling's Formula),$\ln(n!) \approx n \ln n - n$。
    $$ \log_2(n!) \approx n \log_2 n $$
\end{enumerate}

\textbf{结论:} 下界为 $\Omega(n \log n)$。

\section{交易逆序对总数}

\subsection{题目描述}
输入股票记录,若前一天股价高于后一天股价,则构成“交易逆序对”。求逆序对总数。
数据范围:$N \le 50000$。

\subsection{解答与算法}
直接使用双重循环暴力求解的复杂度是 $O(N^2)$,对于 $N=50000$ 会超时。必须使用 $O(N \log N)$ 的解法,通常使用**归并排序(Merge Sort)**的改版。

\textbf{算法思路:}
在归并排序的 `merge` 阶段:
\begin{itemize}
    \item 假设我们将左右两个有序数组 `L` 和 `R` 合并。
    \item 双指针 `i` 指向 `L`,`j` 指向 `R`。
    \item 如果 $L[i] > R[j]$:说明 $L[i]$ 比 $R[j]$ 大。
    \item 关键点:因为 `L` 是有序的,所以 $L[i]$ 之后的所有元素(共 `mid - i + 1` 个)都必定大于 $R[j]$。
    \item 这些都构成了逆序对,将数量加到总数中,并将 $R[j]$ 放入临时数组。
\end{itemize}

\subsection{代码实现 (Python)}

\begin{lstlisting}[language=Python]
class Solution:
    def reversePairs(self, record: list[int]) -> int:
        self.count = 0
        self.merge_sort(record, 0, len(record) - 1)
        return self.count

    def merge_sort(self, nums, left, right):
        if left >= right:
            return
        
        mid = (left + right) // 2
        self.merge_sort(nums, left, mid)
        self.merge_sort(nums, mid + 1, right)
        self.merge(nums, left, mid, right)

    def merge(self, nums, left, mid, right):
        temp = []
        i, j = left, mid + 1
        
        while i <= mid and j <= right:
            if nums[i] <= nums[j]:
                temp.append(nums[i])
                i += 1
            else:
                # 发现逆序对:nums[i] > nums[j]
                # 左边数组中,从 i 到 mid 的所有元素都比 nums[j] 大
                self.count += (mid - i + 1)
                temp.append(nums[j])
                j += 1
        
        # 处理剩余元素
        while i <= mid:
            temp.append(nums[i])
            i += 1
        while j <= right:
            temp.append(nums[j])
            j += 1
            
        # 拷回原数组
        for k in range(len(temp)):
            nums[left + k] = temp[k]
\end{lstlisting}

\end{document}