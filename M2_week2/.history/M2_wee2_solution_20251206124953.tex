\documentclass[a4paper,12pt]{article}
\usepackage[UTF8]{ctex}
\usepackage{amsmath, amssymb, amsthm}
\usepackage{listings}
\usepackage{xcolor}
\usepackage{geometry}
\usepackage{enumitem}
\usepackage{tcolorbox}
\usepackage{titlesec}

\geometry{left=2.5cm, right=2.5cm, top=2.5cm, bottom=2.5cm}

% 定义颜色
\definecolor{codegreen}{rgb}{0,0.6,0}
\definecolor{codegray}{rgb}{0.5,0.5,0.5}
\definecolor{codepurple}{rgb}{0.58,0,0.82}
\definecolor{backcolour}{rgb}{0.95,0.95,0.95}
\definecolor{answercolor}{rgb}{0.1,0.3,0.6}

% 优化代码显示
\lstset{
    language=Python,
    backgroundcolor=\color{backcolour},
    basicstyle=\ttfamily\small,
    keywordstyle=\color{blue}\bfseries,
    commentstyle=\color{codegreen}\itshape,
    stringstyle=\color{codepurple},
    numberstyle=\tiny\color{codegray},
    numbers=left,
    frame=single,
    rulecolor=\color{black!30},
    breaklines=true,
    breakatwhitespace=false,
    tabsize=4,
    showspaces=false,
    showstringspaces=false,
    captionpos=b
}

% 优化列表样式
\setlist[enumerate]{itemsep=0.3em, parsep=0.2em, leftmargin=2em}
\setlist[itemize]{itemsep=0.2em, parsep=0.1em, leftmargin=2em}

% 优化章节标题样式
\titleformat{\subsection}
  {\normalfont\Large\bfseries\color{answercolor}}
  {问题\thesubsection}{1em}{}

\titlespacing*{\subsection}{0pt}{2em}{1em}

\title{\textbf{QTA Week 2 答案汇总}}
\author{}
\date{\today}

\begin{document}

\maketitle
\vspace{-1em}

% ========== 问题一 ==========
\subsection*{问题一}

\noindent\textbf{答案:}
\begin{equation*}
    \mathbb{E}[N_{HH}] = 6, \quad \mathbb{E}[N_{HT}] = 4
\end{equation*}

\noindent\textit{简析:}\\
根据前后缀匹配原理,$N_{HH} = 2^1 + 2^2 = 6$;$N_{HT} = 2^2 = 4$。

\vspace{1em}
\noindent\rule{\textwidth}{0.4pt}

% ========== 问题二 ==========
\subsection*{问题二}

\noindent\textbf{答案:} 18

\noindent\textit{简析:}\\
模式 $HTHH$ 在 $k=1$ (H) 和 $k=4$ (HTHH) 时前后缀匹配。因此:
\begin{equation*}
    \mathbb{E}[N_{HTHH}] = 2^1 + 2^4 = 2 + 16 = 18
\end{equation*}

\vspace{1em}
\noindent\rule{\textwidth}{0.4pt}

% ========== 问题三 ==========
\subsection*{问题三}

\noindent\textbf{答案:} $0.75$ (或 $75\%$)

\noindent\textit{简析:}\\
Bob 的模式为 $TH$。分析如下:
\begin{itemize}
    \item 若第一枚是 T:Bob 必胜(概率 $0.5$)
    \item 若第一枚是 H,第二枚是 T:Bob 胜(概率 $0.5 \times 0.5 = 0.25$)
\end{itemize}
总概率为 $P(\text{Bob 胜}) = 0.5 + 0.25 = 0.75$。

\vspace{1em}
\noindent\rule{\textwidth}{0.4pt}

% ========== 问题四 ==========
\subsection*{问题四}

\noindent\textbf{答案:} 利用 $P(T)=0.4$ 和冯·诺依曼构造法组合

\noindent\textit{算法流程:}
\begin{enumerate}[leftmargin=2.5em]
    \item 抛一次硬币:
    \begin{itemize}
        \item 若为 H(概率 $0.6$):失败,重新开始
        \item 若为 T(概率 $0.4$):进入步骤 2
    \end{itemize}

    \item 连续抛两次硬币,使用冯·诺依曼方法:
    \begin{itemize}
        \item 若结果为 $HT$:返回成功(总概率 $0.4 \times 0.5 = 0.2$)
        \item 若结果为 $TH$:返回失败
        \item 若结果为 $HH$ 或 $TT$:重复步骤 2,直到出现 $HT$ 或 $TH$
    \end{itemize}
\end{enumerate}

\vspace{1em}
\noindent\rule{\textwidth}{0.4pt}

% ========== 问题五 ==========
\subsection*{问题五}

\noindent\textbf{答案:} $498$

\noindent\textit{简析:}\\
考虑数字 $\{0, 0, 1, 2, 3, 4\}$ 的全排列:
\begin{itemize}
    \item 全排列去重后:$606$ 个
    \item 减去以 $0$ 开头的非法数字:$108$ 个
    \item 最终结果:$606 - 108 = 498$
\end{itemize}

\vspace{1em}
\noindent\rule{\textwidth}{0.4pt}

% ========== 问题六 ==========
\subsection*{问题六}

\noindent\textbf{答案:} \textbf{C}

\noindent\textit{简析:}\\
外层装饰器 \texttt{log\_execution} 未使用 \texttt{@wraps(func)},导致被装饰函数的 \texttt{\_\_name\_\_} 和 \texttt{\_\_doc\_\_} 属性被覆盖为装饰器内部函数 \texttt{wrapper\_log} 的属性。

\vspace{1em}
\noindent\rule{\textwidth}{0.4pt}

% ========== 问题七 ==========
\subsection*{问题七}

\noindent\textbf{答案:} Python 动态规划解法

\vspace{0.5em}
\noindent\textit{核心思路:}\\
使用动态规划维护两个状态数组:\texttt{hold}(持有股票)和 \texttt{sold}(已卖出)。考虑冷冻期 $D$ 天的约束。

\begin{lstlisting}
def maxProfit(prices, D, fee):
    n = len(prices)
    hold = [-float('inf')] * n
    sold = [-float('inf')] * n
    max_pre_cool = 0

    for i in range(n):
        if i > D:
            max_pre_cool = max(max_pre_cool, sold[i-D-1])

        # Hold: 昨天持有 vs 今天买入(用解冻资金)
        if i == 0:
            hold[i] = -prices[i]
        else:
            hold[i] = max(hold[i-1], max_pre_cool - prices[i])

        # Sold: 昨天持有 + 今天卖出
        sold[i] = hold[i-1] + prices[i] - fee

    return max(max(sold), 0)
\end{lstlisting}

\noindent\textit{时间复杂度:} $O(n)$;\textit{空间复杂度:} $O(n)$

\end{document}
