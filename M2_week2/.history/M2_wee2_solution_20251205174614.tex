\documentclass[a4paper,12pt]{article}
\usepackage[UTF8]{ctex}
\usepackage{amsmath, amssymb}
\usepackage{listings}
\usepackage{xcolor}

\lstset{
    language=Python,
    basicstyle=\ttfamily\small,
    keywordstyle=\color{blue},
    frame=single,
    breaklines=true
}

\begin{document}

\section*{QTA Week 2 答案汇总}

[cite_start]\subsection*{问题一 [cite: 5-9]}
\textbf{答案:} $\mathbb{E}[N_{HH}] = 6$, $\mathbb{E}[N_{HT}] = 4$。
\\ \textit{简析:} $N_{HH} = 2^1 + 2^2 = 6$; $N_{HT} = 2^2 = 4$。

[cite_start]\subsection*{问题二 [cite: 10-18]}
\textbf{答案:} 18
\\ \textit{简析:} 模式 $HTHH$ 在 $k=1$ (H) 和 $k=4$ (HTHH) 时前后缀匹配。
$$ \mathbb{E} = 2^1 + 2^4 = 2 + 16 = 18 $$

[cite_start]\subsection*{问题三 [cite: 20-25]}
\textbf{答案:} 0.75 (或 75\%)
\\ \textit{简析:} Bob ($TH$) 只要第一枚是 T 则必胜 (0.5);若第一枚是 H,只有第二枚是 T 时 Bob 才能赢 (0.25)。总概率 $0.5 + 0.25 = 0.75$。

[cite_start]\subsection*{问题四 [cite: 26-27]}
\textbf{答案:} 利用 $P(T)=0.4$ 和冯·诺依曼构造法 ($P=0.5$) 组合。
\\ \textit{算法:}
\begin{enumerate}
    \item 抛一次硬币,若为 H (0.6) 则失败。
    \item 若为 T (0.4),则连续抛两次硬币:
    \begin{itemize}
        \item 若 $HT$:返回成功 (总概率 $0.4 \times 0.5 = 0.2$)。
        \item 若 $TH$:返回失败。
        \item 若 $HH$ 或 $TT$:重复步骤2直到出现 $HT$ 或 $TH$。
    \end{itemize}
\end{enumerate}

[cite_start]\subsection*{问题五 [cite: 28-29]}
\textbf{答案:} 498
\\ \textit{简析:} 全排列去重后为 606,减去以 0 开头的非法数字 108,得 498。

[cite_start]\subsection*{问题六 [cite: 30-71]}
\textbf{答案:} C
\\ \textit{简析:} 外层装饰器 \texttt{log\_execution} 未使用 \texttt{@wraps},导致函数名和文档变为装饰器内部函数 \texttt{wrapper\_log} 的属性。

[cite_start]\subsection*{问题七 [cite: 72-106]}
\textbf{答案:} Python 动态规划解法
\begin{lstlisting}
def maxProfit(prices, D, fee):
    n = len(prices)
    hold = [-float('inf')] * n
    sold = [-float('inf')] * n
    max_pre_cool = 0
    
    for i in range(n):
        if i > D:
            max_pre_cool = max(max_pre_cool, sold[i-D-1])
        
        # Hold: 昨天持有 vs 今天买入(用解冻资金)
        if i == 0: hold[i] = -prices[i]
        else:      hold[i] = max(hold[i-1], max_pre_cool - prices[i])
            
        # Sold: 昨天持有 + 今天卖出
        sold[i] = hold[i-1] + prices[i] - fee
        
    return max(max(sold), 0)
\end{lstlisting}

\end{document}